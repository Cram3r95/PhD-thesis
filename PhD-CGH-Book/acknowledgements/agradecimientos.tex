%%%%%%%%%%%%%%%%%%%%%%%%%%%%%%%%%%%%%%%%%%%%%%%%%%%%%%%%%%%%%%%%%%%%%%%%%%%
%
% Generic template for TFC/TFM/TFG/Tesis
%
% $Id: agradecimientos.tex,v 1.7 2015/06/05 00:10:31 macias Exp $
%
% By:
%  + Javier Macías-Guarasa. 
%    Departamento de Electrónica
%    Universidad de Alcalá
%  + Roberto Barra-Chicote. 
%    Departamento de Ingeniería Electrónica
%    Universidad Politécnica de Madrid   
% 
% Based on original sources by Roberto Barra, Manuel Ocaña, Jesús Nuevo,
% Pedro Revenga, Fernando Herránz and Noelia Hernández. Thanks a lot to
% all of them, and to the many anonymous contributors found (thanks to
% google) that provided help in setting all this up.
%
% See also the additionalContributors.txt file to check the name of
% additional contributors to this work.
%
% If you think you can add pieces of relevant/useful examples,
% improvements, please contact us at (macias@depeca.uah.es)
%
% You can freely use this template and please contribute with
% comments or suggestions!!!
%
%%%%%%%%%%%%%%%%%%%%%%%%%%%%%%%%%%%%%%%%%%%%%%%%%%%%%%%%%%%%%%%%%%%%%%%%%%%

% Use this if you don't like the fancy style
\thispagestyle{empty}

\ifthenelse{\equal{\myLanguage}{english}}
{
  \chapter*{Acknowledgements}
  \label{cha:acknowledgements}
  \markboth{Acknowledgements}{Acknowledgements}
  \addcontentsline{toc}{chapter}{Acknowledgements}
}
{
  \chapter*{Agradecimientos}
  \label{cha:agradecimientos}
  \markboth{Agradecimientos}{Agradecimientos}
  \addcontentsline{toc}{chapter}{Agradecimientos}
}

% ``Más vale un minuto de ilusión que mil horas de
% razonamiento''... (cortesía de Roberto Barra)

Esta Tesis Doctoral supone el culmen a cuatro años (Abril 2019 - Abril 2023) realmente duros, cargado de emociones, triunfos, pandemias, estafas y tropiezos, todo a partes iguales. Este es probablemente (aunque como diría Sean Connery interpretando a James Bond en 1983, \textit{Never Say Never Again}) mi último gran documento individual, académicamente hablando.

Durante mi etapa universitaria (2013 hasta el momento, 2023) he tenido ciertos momentos puntuales en los que he sentido un salto cualitativo como profesional: El primero fue en el segundo cuatrimestre de segundo de carrera, cuando las cosas se pusieron tensas con Control II e Informática Industrial. Vaya sudores. El segundo probablemente fue con el fallecimiento de mi madre durante mi ERASMUS+ en Irlanda. Duros y oscuros momentos, alejado de mis seres queridos. El tercer momento llega en segundo de máster, durante mi querido ERASMUS+ en Finlandia, donde compagino una estancia preciosa en Tampere con el máster y un pre-inicio de doctorado. Me equivoqué al empezar tan pronto con la beca, "queriendo cobrar" cuanto antes, en vez de terminar tranquilamente el TFM y plantear la tesis, pero eso no lo sabría hasta tiempo después. Pero no es hasta la tesis donde empezaron los quebraderos de cabeza reales. Continuamente altibajos, mala planificación por mi parte, momentos puntuales donde me equivoqué rotundamente al empecinarme en soldar una estructura compleja para nuestro vehículo sin ayuda, no estudiar PyTorch tras el congreso WAF 2018 tras la sugerencia de mi tutor, no enfocarme en técnica individual hasta bien entrado el doctorado, no querer hacer nada hasta que no tuviese la teoría perfectamente asimilada, tener demasiado respeto a la Inteligencia Artificial y escurrir el bulto de mi tesis en un compañero mientras yo me dedicaba a integrar y corregir los bugs del grupo que para mí \textit{era lo fácil}. Mal. Todo mal. Pero todo cambió tras mi segunda estancia, en Estados Unidos, cuando tras llorar por no entender el camino a seguir, nadie que me ayudara, decidí crear mi propio camino, con paciencia y fé, práctica y error compaginado con lectura de artículos, para mejorar mi confianza y autoestima, y finalmente logré empezar a entender lo que era el Deep Learning. Gracias a todos mis errores, desventuras y discusiones, a día de hoy, excepto momentos inevitables, me encuentro con muchísima capacidad para atacar y gestionar prácticamente cualquier problema, consultar documentación y organizarme, aunque esta sigue siendo mi tarea pendiente.

Cada año, desde hace ya varios, mi primera publicación en Instagram viene seguida de la frase "Trabaja duro en silencio y deja que tu éxito haga todo el ruido". Filosofía Kaizen, de mejora y aprendizaje continuo, para así cada día entender el mundo un poquito mejor. Si toda la dedicación y estudio que he depositado en este trabajo sirven para algo en mi futuro, sé que todo el esfuerzo habrá merecido la pena.

Después de este particular monólogo, a lo cual soy muy propenso y de lo cual mis amigos y compañeros no cesan en su empeño de recordádmelo, debo, como no puede ser de otra manera, dar paso a los agradecimientos.

En primer lugar, me gustaría agradecer a mis profesores del grupo RobeSafe, especialmente a mis tutores Luis Miguel Bergasa Pascual y Rafael Barea Navarro, por ofrecerme estar en el grupo (así como aguantarme) durante todos estos años e intentar que tuviésemos el mejor \textit{experimental setup} y \textit{roadmap} en el laboratorio, aunque no fuese siempre sencillo. Sin dudas considero realmente interesante la temática propuesta en esta tesis doctoral, predicción de agentes en el contexto de vehículo inteligente, ya que entra en el plano filosófico sobre cómo razonar el futuro de los objetos y cómo podría afectar a la capacidad ejecutiva del agente que deba tomar una decisión. Mi mente hace tiempo que cambió y me fijo siempre que conduzco de todo lo que intento reproducir con mis estudios. Habrá que seguir esta tendencia muy de cerca en los próximos años, porque personalmente considero que sus aplicaciones son fantásticas.

A mis tutores en las estancias de doctorado, Christoph Stiller y Eduardo Molinos en el Karslruhe Institute of Technology (KIT, Alemania) y Wei Zhan y Masayoshi Tomizuka en la University of California, Berkeley (UCB, Estados Unidos). Se suele decir que unas veces se gana y otras se aprende, y yo en estas estancias quizás aprendí demasiado ... No obstante, me guardo grandísimos momentos (admirar las secuoyas gigantes o hacer mi primera escalada en roca entre ellos) y amigos, como Su Shaoshu o Frank Bieder, con los que aún guardo un cierto contacto. 

A mis compañeros, mejor dicho, amigos, de laboratorio: Javier Araluce, Rodrigo, Felipe (chavalín), Santiago, Miguel Antunes, Miguel Eduardo,
antiguos compañeros como Javier del Egido, Óscar, Alejandro, Eduardo, Roberto y Pablo Alcantarilla, y a nuevos becarios como Fabio, Navil y Pablo. Gracias de corazón por estar ahí, en nuestras charlas sobre tecnología, empleos, el camino correcto a seguir y la vida en general.

Especial mención vuelvo a hacer a Miguel Eduardo y mi compañero Marcos Conde, cuya apoyo intenso ayudó a centrar mi camino en técnica y escritura. Muchísimas gracias por todo lo que aprendí a vuestro lado. Especial mención a mi querido profesor Ángel Álvarez, por sus valiosos consejos para afrontar mi carrera profesional y mi vida en general de la mejor forma posible. Eres muy grande.

A mis amigos de la universidad, especialmente a Rocío, Juan Carlos, Esther, Sergio, Pablo, Rubén y Adrián Rocandio. Aún me acuerdo de cuando empezamos con la carrera y como el paso inexorable del tiempo nos moldea a conveniencia. Os deseo lo mejor en vuestro futuro. 

A mis buenos amigos Samuel y Adrián, con quien gran parte de mi vida he compartido. Con especial cariño guardo las interminables charlas sobre la vida y el futuro después de Karate, de comer, de cenar, en el coche, siempre quejándonos de la hora que marcaba el reloj al final de tan interminables conversaciones. 

A mi familia, uno de los pilares de mi vida. A mi padre Juan Antonio y a mi madre Petra, que en paz descanse, les debo todo lo que soy y es por ello por lo que les estaré siempre agradecido. Querido padre, gracias por ser tan Genaro, arisco y pesado. Siempre has sido mi ejemplo a seguir, aunque cuando tenga 42 años me sigas regañando por subir con las zapatillas puestas. Querida madre, no se muere quien se va, sólo se muere el que se olvida, y tú nunca caerás en el olvido. A mi querida hermana-calili-chessmaster Silvia, con quien tantas regañinas he tenido, pero el cariño que nos tenemos las supera a todas. A mi perrita Nuka (a.k.a. Dragón, ChupaChups cuando le cortamos el pelo o Nuki-Nuki), cuyos paseos matutinos son probablemente el ingrediente secreto para la elaboración de esta tesis, dando rienda suelta a mi cabeza para imaginar nuevas propuestas mientras miraba el cielo azul. A mi querido \textit{experimental setup} que me ha acompañado durante toda mi vida académica: silla de madera de la cocina, ratón de Hello Kitty tomado prestado de mi hermana y mi querido portátil táctil. Sois la base de todo mi trabajo. 

Al resto de la familia, amigos, compañeros, entrenadores y profesores, gracias por todo.

Y, por último, la persona más importante de mi vida ahora mismo. Mi querida Marta, la persona más maravillosa y buena que conozco. Hemos compartido risas, lloros, besos y abrazos. Nunca me cansaré de repetirte lo suave que tienes la piel tras darte un beso en la mejilla y después hacerte de rabiar. Espero que esta situación esté dentro de un while cuya condición sea \textit{True}. 

\begin{center}
\textit{"Te quiero más que ayer, pero menos que mañana. Hoy, y siempre"}
\end{center}

Mi querido lector, disculpa mi monólogo de agradecimientos, es mi forma de ser y la cual tengo por bandera, aunque creo que ha quedado bonito. Podría decir mil anécdotas más de mi doctorado, pero como diría Aragorn, legítimo Rey de Gondor, enfrente de la mismísima Puerta Negra: \textit{Hoy no es ese día}. 

Vamos a la lectura importante, que empiece el \textit{Rock and Roll} !!

% Back to normal JIC. Use it if you set \pagestyle{myplain} above
%\pagestyle{fancy}

%%% Local Variables:
%%% TeX-master: "../book"
%%% End: