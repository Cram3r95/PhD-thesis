%%%%%%%%%%%%%%%%%%%%%%%%%%%%%%%%%%%%%%%%%%%%%%%%%%%%%%%%%%%%%%%%%%%%%%%%%%%
%
% Generic template for TFC/TFM/TFG/Tesis
%
% $Id: agradecimientos.tex,v 1.7 2015/06/05 00:10:31 macias Exp $
%
% By:
%  + Javier Macías-Guarasa. 
%    Departamento de Electrónica
%    Universidad de Alcalá
%  + Roberto Barra-Chicote. 
%    Departamento de Ingeniería Electrónica
%    Universidad Politécnica de Madrid   
% 
% Based on original sources by Roberto Barra, Manuel Ocaña, Jesús Nuevo,
% Pedro Revenga, Fernando Herránz and Noelia Hernández. Thanks a lot to
% all of them, and to the many anonymous contributors found (thanks to
% google) that provided help in setting all this up.
%
% See also the additionalContributors.txt file to check the name of
% additional contributors to this work.
%
% If you think you can add pieces of relevant/useful examples,
% improvements, please contact us at (macias@depeca.uah.es)
%
% You can freely use this template and please contribute with
% comments or suggestions!!!
%
%%%%%%%%%%%%%%%%%%%%%%%%%%%%%%%%%%%%%%%%%%%%%%%%%%%%%%%%%%%%%%%%%%%%%%%%%%%

% Use this if you don't like the fancy style
\thispagestyle{empty}

\ifthenelse{\equal{\myLanguage}{english}}
{
  \chapter*{Acknowledgements}
  \label{cha:acknowledgements}
  \markboth{Acknowledgements}{Acknowledgements}
  \addcontentsline{toc}{chapter}{Acknowledgements}
}
{
  \chapter*{Agradecimientos}
  \label{cha:agradecimientos}
  \markboth{Agradecimientos}{Agradecimientos}
  \addcontentsline{toc}{chapter}{Agradecimientos}
}

% ``Más vale un minuto de ilusión que mil horas de
% razonamiento''... (cortesía de Roberto Barra)

Esta Tesis Doctoral supone el culmen a cuatro años (Abril 2019 - Septiembre 2023) realmente duros, cargado de emociones, triunfos, pandemias, estafas y tropiezos, todo a partes iguales. Este es probablemente (aunque como diría Sean Connery interpretando a James Bond en 1983, \textit{Never Say Never Again}) mi último gran documento individual, académicamente hablando.

Durante mi etapa universitaria (2013 hasta el momento, 2023) he tenido ciertos momentos puntuales en los que he sentido un salto cuantitativo a nivel profesional: El primero fue en el segundo cuatrimestre de segundo de carrera, cuando las cosas se pusieron tensas con Control II e Informática Industrial. Vaya sudores. El segundo probablemente fue con el fallecimiento de mi madre durante mi ERASMUS+ en Irlanda. Duros y oscuros momentos, alejado de mis seres queridos, que afronté con la mayor entereza posible con la ayuda de compañeros que conocí allí, apoyo de familiares y amigos y cuando mi padre y mi hermana me vinieron a visitar, pero sobre todo con la regla de las tres C: Cabeza, Corazón y Coj****. Ya me entendéis. El tercer momento llega en segundo de máster, durante mi querido ERASMUS+ en Finlandia, donde compagino una estancia preciosa en Tampere (visitando Laponia, Rusia (nunca olvidaré cuando cogimos un piso tan barato en Moscú, \ie \ 7 euros la noche, que había chinches en la ropa al día siguiente), Estonia, Letonia, etc) y un pre-inicio de doctorado. Me equivoqué al empezar tan pronto con la beca de doctorado FPI, "queriendo cobrar" cuanto antes, en vez de terminar tranquilamente el TFM y plantear el doctorado, pero eso no lo sabría hasta tiempo después. 

Fue en la tesis cuando empezaron realmente los quebraderos de cabeza. Continuamente altibajos, mala planificación por mi parte (y no tan mi parte ...), momentos puntuales donde me equivoqué rotundamente al querer soldar una estructura compleja para nuestro vehículo eléctrico sin ninguna ayuda, no estudiar PyTorch tras el congreso WAF 2018 tras la sugerencia de mi tutor, no enfocarme en técnica individual hasta bien entrado el doctorado, no querer hacer nada hasta que no tuviese la teoría perfectamente asimilada, tener demasiado respeto a la Inteligencia Artificial y escurrir el bulto de mi tesis en un compañero mientras yo me dedicaba a integrar y corregir los \textit{bugs} del grupo que para mí \textit{era lo fácil}. Mal. Todo mal. Pero todo cambió tras mi segunda estancia de doctorado, en Estados Unidos, cuando tras llorar por no entender el camino a seguir (probablemente el mayor reto de un doctorado), nadie que me ayudara, decidí crear mi propio camino, con paciencia y fé, práctica y error, todo ello compaginado con lectura de artículos, para mejorar mi confianza y autoestima y así afrontar con el mayor estándar de calidad posible el tiempo que me quedaba de doctorado. Gracias a todos mis errores, desventuras y discusiones, a día de hoy, excepto momentos inevitables, me encuentro con muchísima más capacidad para atacar, diseccionar y gestionar prácticamente cualquier problema, consultar documentación y organizarme, aunque esta última sigue siendo mi tarea pendiente. En ese sentido, me gustaría recalcar una frase recientemente dicha (Julio, 2023) por parte de Petar Veličković, uno de los mayores expertos actuales en \textit{Graph Neural Networks} a nivel mundial, resumiendo lo que representa terminar un doctorado:

\begin{center}
	\textit{"A successful PhD is not something just that makes you a very deep expert in a particular area, but it also makes you a master adapter"}
\end{center}

Desde hace ya varios, mi primera publicación del año en Instagram viene seguida de la frase "Trabaja duro en silencio y deja que tu éxito haga todo el ruido". Filosofía Kaizen, mejora y aprendizaje continuo, para así cada día entender el mundo un poquito mejor. Si toda la dedicación y estudio que he depositado en este trabajo sirven para algo en el futuro, sé que todo el esfuerzo habrá merecido la pena.

Después de este particular monólogo, a lo cual soy muy propenso y de lo cual mis amigos y compañeros no cesan en su empeño de recordádmelo, debo, como no puede ser de otra manera, dar paso a los agradecimientos.

En primer lugar, me gustaría agradecer a mis profesores del grupo RobeSafe, especialmente a mis tutores Luis Miguel Bergasa Pascual y Rafael Barea Navarro (siempre recordaré el abrigo que me prestaste cuando fuimos a Nueva Zelanda en 2019 porque me había dejado el mío en el aeropuerto), por ofrecerme estar en el grupo (así como aguantarme) durante todos estos años e intentar que tuviésemos el mejor \textit{experimental setup} y \textit{roadmap} en el laboratorio, aunque no fuese siempre sencillo. Especialmente me gustaría resaltar de nuevo la figura de Luis Miguel, a quien con mucho cariño considero mi padre académico, por confiar en mí en su momento, sus múltiples consejos sobre la vida y el trabajo, y haber lidiado con mi ambición y frustración a partes iguales como un gran capitán que ha sabido guiar una humilde embarcación para que creciera personal y profesionalmente con el paso inexorable del tiempo. Sin lugar a dudas considero realmente interesante la temática propuesta en esta tesis doctoral, predicción de movimiento en el contexto de vehículo inteligente, ya que entra en el plano filosófico sobre cómo razonar el futuro de la escena de tráfico para tener una mayor seguridad y comfort durante la conducción. Es bonito ver como durante el doctorado tu mente cambia y te fijas en tu día a día (en este caso, fundamentalmente a la hora de conducir) en todo lo que intentas reproducir durante tu investigación. Habrá que seguir esta tendencia muy de cerca en los próximos años, porque personalmente considero que sus aplicaciones son fantásticas.

A mis tutores en las estancias de doctorado, Christoph Stiller y Eduardo Molinos en el Karslruhe Institute of Technology (KIT, Alemania) y Wei Zhan y Masayoshi Tomizuka en la University of California, Berkeley (UCB, Estados Unidos). Se suele decir que unas veces se gana y otras se aprende, y yo en estas estancias quizás aprendí demasiado, porque vaya tela ... No obstante, me guardo momentos muy bonitos (admirar las secuoyas gigantes de Yosemite, visitar más solo que la una la zona de Baden-Wurtemberg (la Selva Negra del Age of Empires!), hacer mi primera escalada en roca, las milanesas que le quitaba (cariñosamente) a los argentinos en Berkeley durante mis incursiones nocturnas a la nevera, comer guisantes con jamón junto al río con la luz tenue del atardecer con una tarjeta de crédito porque se me había olvidado el tenedor al visitar Heidelberg, etc.) y buenos amigos, como Su Shaoshu o Frank Bieder, con los que aún guardo un cierto contacto. 

A mis compañeros, mejor dicho, amigos, de laboratorio, especialmente Javier Araluce, Rodrigo, Felipe (chavalín), Santiago, Miguel Antunes, Miguel Eduardo, antiguos compañeros como Javier del Egido, Óscar, Alejandro, Eduardo, Roberto y Pablo Alcantarilla, y a nuevos compañeros del grupo como Fabio, Navil y Pablo. Gracias de corazón por estar ahí, por vuestras charlas sobre tecnología, empleos, el camino correcto a seguir y la vida en general.

Especial mención vuelvo a hacer a Miguel Eduardo y mi compañero Marcos Conde, cuyo apoyo intenso ayudó a centrar mi camino en técnica y escritura. Sois dos pedazo de máquinas. Muchísimas gracias por todo lo que aprendí a vuestro lado y os deseo lo mejor en vuestro futuro profesional y personal.

A mi querido profesor Ángel Álvarez, por sus valiosos consejos para afrontar mi carrera profesional y mi vida en general de la mejor forma posible. Eres muy grande.

A mis grandes amigos de la universidad con los que sigo teniendo un contacto bastante estrecho y les quiero un montón, especialmente a Rocío, Juan Carlos, Esther, Sergio, Antonio, Pablo, Rubén y Adrián Rocandio. Una alegría cuando me junto con ellos siempre y compartimos recuerdos y objetivos futuros, tanto personales como profesionales. Aún me acuerdo de cuando empezamos con la carrera y como el paso inexorable del tiempo nos moldea a conveniencia. Os deseo lo mejor.

A mis buenos amigos Samuel y Adrián, con quien gran parte de mi vida he compartido. Con especial cariño guardo las interminables charlas sobre la vida y el futuro después de entrenar en Karate, de comer, de cenar, en el coche, siempre quejándonos de la hora que marcaba el reloj al final de tan extenuantes conversaciones. 

A mi familia, uno de los pilares de mi vida. A mis abuelos Gerardo, Petra y Juan Pablo, que están en el cielo, y a mi querida abuela Juli, con la que tantas porras con chocolate he comido de pequeño y no tan pequeño, siendo la huelga semanal de dos humildes euros mi mayor tesoro hace no tantos años. A mi padre Juan Antonio y a mi madre Petra, que en paz descanse, les debo todo lo que soy y es por ello por lo que les estaré siempre agradecido. Querido padre, gracias por ser tan Genaro, arisco y pesado. Siempre has sido mi ejemplo a seguir, aunque cuando tenga 42 años me sigas regañando por subir con las zapatillas puestas a la habitación o no cerrar la puerta porque entran moscas. Querida madre, no se muere quien se va, sólo se muere el que se olvida, y tú nunca caerás en el olvido. A mi querida hermana-calili-\textit{chessmaster} Silvia, con quien tantas regañinas he tenido, pero el cariño que nos tenemos las supera a todas. A mi perrita Nuka (\aka \ Dragón, ChupaChups cuando le cortamos el pelo o Nuki-Nuki), cuyos paseos matutinos son probablemente el ingrediente secreto para la elaboración de este documento, dando rienda suelta a mi cabeza para imaginar nuevas propuestas mientras miraba el cielo azul. 

A mi querido \textit{experimental setup} que me ha acompañado durante toda mi vida académica: Silla de madera de la cocina, ratón de Hello Kitty "tomado prestado" de mi hermana, mi querido portátil táctil, disco duro portátil con fundita de cartón y goma, y la mochila negra que me lleva acompañando en mis aventuras desde Irlanda. Sois la base de todo mi trabajo y habéis recorrido medio mundo, compañeros (menos la silla, claro)!

Al resto de la familia, amigos, compañeros, entrenadores y profesores, gracias por todo.

Y, por último, la persona más importante de mi vida ahora mismo. Mi querida Marta, la persona más maravillosa y buena que conozco. Hemos compartido risas, lloros, besos y abrazos. Nunca me cansaré de repetirte lo suave que tienes la piel tras darte un beso en la mejilla y después hacerte de rabiar. Espero que esta situación esté dentro de un bucle \textit{while} cuya condición sea \textit{True}. 

\begin{center}
\textit{"Te quiero más que ayer, pero menos que mañana. Hoy, y siempre ..."}
\end{center}

Mi querido lector, disculpa mi monólogo (te lo avisé) de agradecimientos, es mi forma de ser y la cual tengo por bandera, aunque creo que ha quedado bonito. Podría decir mil y un anécdotas más de mi doctorado, pero como diría Aragorn, hijo de Arathorn, heredero de Isildur y legítimo Rey de Gondor, enfrente de la mismísima Puerta Negra de Mordor: \textit{Hoy no es ese día}. 

Vamos a la lectura importante, que empiece el \textit{Rock and Roll} !!

% Back to normal JIC. Use it if you set \pagestyle{myplain} above
%\pagestyle{fancy}

%%% Local Variables:
%%% TeX-master: "../book"
%%% End: