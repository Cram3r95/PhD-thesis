%%%%%%%%%%%%%%%%%%%%%%%%%%%%%%%%%%%%%%%%%%%%%%%%%%%%%%%%%%%%%%%%%%%%%%%%%%% 
% 
% Generic template for TFC/TFM/TFG/Tesis
% 
% By:
% + Javier Macías-Guarasa. 
% Departamento de Electrónica
% Universidad de Alcalá
% + Roberto Barra-Chicote. 
% Departamento de Ingeniería Electrónica
% Universidad Politécnica de Madrid   
% 
% Based on original sources by Roberto Barra, Manuel Ocaña, Jesús Nuevo,
% Pedro Revenga, Fernando Herránz and Noelia Hernández. Thanks a lot to
% all of them, and to the many anonymous contributors found (thanks to
% google) that provided help in setting all this up.
% 
% See also the additionalContributors.txt file to check the name of
% additional contributors to this work.
% 
% If you think you can add pieces of relevant/useful examples,
% improvements, please contact us at (macias@depeca.uah.es)
% 
% You can freely use this template and please contribute with
% comments or suggestions!!!
% 
%%%%%%%%%%%%%%%%%%%%%%%%%%%%%%%%%%%%%%%%%%%%%%%%%%%%%%%%%%%%%%%%%%%%%%%%%%% 

\chapter{Conclusions and Future Works}
\label{cha:conclusions_and_future_works}

\begin{FraseCelebre}
	\begin{Frase}
		El mundo no es todo alegría y color, es un lugar terrible y por muy duro que seas es capaz de arrodillarte a golpes y tenerte sometido a golpes permanente si no se lo impides; Ni tú ni yo ni nadie golpea mas fuerte que la vida. Pero no importa lo fuerte que golpeas, sino lo fuerte que pueden golpearte y los aguantas mientras avanzas, hay que soportar sin dejar de avanzar.
		
		¡Así es como se gana!
		
		Si tú sabes lo que vales, vé y consigue lo que mereces pero tendrás que soportar los golpes y no puedes estar diciendo que no estás donde querías llegar por culpa de él o de ella, eso lo hacen los cobardes y tú no lo eres.
		
		TÚ ERES CAPAZ DE TODO.
	\end{Frase}
	\begin{Fuente}
		Discurso de Rocky a su hijo \\
		Rocky Balboa
	\end{Fuente}
\end{FraseCelebre}

\section{Conclusions}
\label{sec:6_introduction}

In this thesis, a series of interaction-aware trajectory prediction methods, including single-agent trajectory prediction, multi-agent trajectory prediction, and multimodal trajectory prediction, are developed for autonomous driving. Besides, the
impacts of trajectory prediction on trajectory planning are also investigated.
In Chapter 3, a novel framework with consideration of vehicle-infrastructure heterogeneous interactions is proposed for trajectory prediction of a single target vehicle. In the proposed scheme, a heterogeneous graph is developed to represent
the interactions, where the nodes contain features extracted from corresponding
encoders. Besides, a novel heterogeneous graph social pooling (HGS) module is
designed to extract high-level interaction features. The framework can be easily expanded for highway driving scenarios. Experimental results obtained using
real-world driving datasets show that the proposed HGS method outperforms existing interaction-aware methods in terms of prediction accuracy. Besides, ablative
studies demonstrate that the consideration of vehicle-infrastructure heterogeneous
interactions effectively improves the prediction accuracy compared to those methods only considering inter-vehicle interactions.
Then, the above prediction method for single-agent is generalized and expanded
for heterogeneous multi-agent trajectory prediction in Chapter 4. To do this, a
novel three-channel framework is designed to jointly consider traffc participants’
dynamics, interaction, and map features. The driving scene is represented in a hybrid way, where the inter-agent interaction in the traffc system is represented with
an edge-featured heterogeneous graph, and the shared local map is represented
with a Bird’s Eye View (BEV) image. Two shared Recurrent Neural Networks
(RNNs) are adopted to capture vehicles’ and pedestrians’ dynamics features from
their historical states, respectively. A novel heterogeneous edge-enhanced graph
attention network (HEAT) is proposed to model the inter-agent interactions, and
a map-sharing technique based on the gate mechanism is also leveraged to share the
local map across all target agents. Experimental validations on real-world driving
datasets of both urban and highway scenarios show that the proposed method not
only achieves state-of-the-art performance but also can provide simultaneous predictions of multi-agent trajectories for a variable number of heterogeneous agents.
Besides the unimodal predictions for single and multiple agents, this thesis also
tackles the inherent multimodality problem of driving behaviors for prediction in
Chapter 5. A novel map-adaptive multimodal trajectory prediction framework is
proposed. Within this framework, through a single graph operation, a variable
number of map-compliant trajectories and a non-map-compliant trajectory can be
generated. Map-compliant predictions are conditioned on either a single candidate
centerline (CCL) or a bunch of all CCLs, making the predictor adaptive to different
road structures. The non-map-compliant prediction captures the irrational driving
behavior for safety concerns. The driving scene is represented with a heterogeneous
hierarchical graph containing both agents and their CCLs. A hierarchical graph
operator (HGO) with an edge-masking technology is proposed to encode the driving
scene. Validation on the Argoverse motion forecasting benchmark shows that the
proposed method achieves state-of-the-art performance with the advantage of mapadaptive capacity.
Beyond pure prediction, in Chapter 6, predictive planning and the impacts of prediction on downstream trajectory planning are also investigated. An interactionaware predictive planner, which is trained to imitate human driving behaviors,
is designed to investigate the problem of how prediction would affect the performance of motion planning. The predictive planner is obtained by training an
oracle planner, which is aware of target agents’ ground truth future trajectories,
and replacing the ground truth with the predicted trajectories for inference during
implementation. Experimental results on a real-world dataset show that the proposed predictive planner achieves better performance over other baselines in terms
of displacement error, miss rate, and collision rate. The gap between the predictive
planner and the oracle planner shows that it is promising to further enhance the
planning performance by improving the prediction accuracy.
To be implemented in real-world self-driving systems, the proposed methods require upstream localization, perception, and tracking results, since the historical
states of other traffc participants are needed as the input of the proposed methods. Perception can be realized using either or both of camera and LiDAR. Other
sensors, such as radar, can also be used for better perception via sensor fusion. For
the single-agent and multi-agent prediction methods in Chapter 3 and Chapter 4,
we are using BEV maps, where only the map is needed. We do not need a BEV
image to show the real-time tra!c. The map can be obtained from main-stream
map providers and converted into images for the usage of our method. The mapadaptive multimodal method in Chapter 5, however, requires a high-definition map
(HD map) of the local area since we need the candidate centerlines of vehicles of interest. The methods can run on both CPUs or GPUs, and using GPUs is suggested
for faster inference.

\section{Future Works}
\label{sec:6_future_works}

Although many studies have been done in trajectory prediction and path planning
in the past years, there are still many aspects that need to be further investigated
in the future.
Scene representation and encoding. Researchers have proposed many methods
to encode driving scenes with different representations. However, there is no unified
representation of various driving scenes so far. The lack of a universal representation limits the generalizability of prediction methods with large-scale deployment
in autonomous vehicles in the real world because a method can hardly be applied
to a situation that cannot be described. Among many representation approaches
proposed so far, graph-based representations are promising because a graph can
accommodate an arbitrary number of heterogeneous objects and represent their
interdependencies via directed edges. For example, when modeling a driving scene
in the context of traffc systems, a node can represent a vehicle, a pedestrian, a
lanelet, a junction, a traffc signal, etc. A new object can always be added to the
existing graph. There are three important steps that need to be done to further
improve the graph-based scene representation and encoding in the future. The first
aspect is to construct the graph with proper connections, that is, to determine the
edge set of the graph. This step needs to identify interdependencies between pairs
of nodes and connect nodes with directed edges for information flow in the graph.
Once the graph structure is settled, the second step is to assign the node, and edge
features properly. This requires researchers to select or design proper encoders
for different kinds of nodes and edges. Then the third step is required to design
graph operators to handle the heterogeneity in the scene graphs. In this step, the
advances in heterogeneous graph neural networks can be leveraged.
Trajectory decoding. For future work, an immediate step is to generalize the
map-adaptive multimodal prediction method proposed in Chapter 5 with uncertainty estimations. The uncertainty includes both motion and mode uncertainties.
The motion uncertainty captures the distribution of agents’ position over a planar
map at each time step, and the mode uncertainty captures the possibility of driving
modalities. The former can be modeled via bivariate Gaussian distributions, and
the latter can be treated as a multi-class classification problem over a variable number of modalities. Then the next step can focus on generalizing uncertainty-aware
multimodal predictions to multi-agent settings by modeling the joint distribution
of multiple agents’ behaviors. Social consistency should be considered in this step
such that there is no conflict between any pair of trajectories in a joint modality
in normal cases. Besides, trajectory predictors should be designed from the ego
vehicle’s point of view. One possible way is to design a decoder that can output
trajectories upon the ego’s request. For example, the predictor can focus on a small
set of target agents requested by the ego vehicle rather than all the agents in sight.
For a specific target agent, the predictor can focus on predicting its driving options
that may affect the ego’s planned trajectories. This attentive approach can reduce
computation efforts for real-time implementations.
Predictive planning. The ultimate goal of trajectory prediction is to further
improve the performance of decision-making and motion control of autonomous
vehicles with respect to safety, smartness, and effciency. So prediction must be integrated into the planning module, and therefore predictive planning is worthwhile
exploring. There are many problems that should be addressed for the development
of predictive planners. First, predictive planners should be able to address prediction uncertainty since prediction can never be exactly the same as the ground truth.
Second, the relationship between prediction and planning needs to be further studied in order to answer the following questions: 1) How would the improvement in
prediction affect the downstream planning performance? 2) Is there a floor of prediction error below which improving prediction accuracy leads to no improvement
or even a negative effect on planning? 3) Can we design a predictive planner that
is scalable to predictors of different uncertainties as long as these uncertainties are
known? Third, learning-based motion planners should be further investigated since
they have great potential to be incorporated with data-driven predictions. However, the current limitations in explainability and reliability need to be addressed.
In general, plenty of effort is needed in these research areas.