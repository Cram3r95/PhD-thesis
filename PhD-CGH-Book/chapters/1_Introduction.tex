%%%%%%%%%%%%%%%%%%%%%%%%%%%%%%%%%%%%%%%%%%%%%%%%%%%%%%%%%%%%%%%%%%%%%%%%%%% 
% 
% Generic template for TFC/TFM/TFG/Tesis
% 
% By:
% + Javier Macías-Guarasa. 
% Departamento de Electrónica
% Universidad de Alcalá
% + Roberto Barra-Chicote. 
% Departamento de Ingeniería Electrónica
% Universidad Politécnica de Madrid   
% 
% Based on original sources by Roberto Barra, Manuel Ocaña, Jesús Nuevo,
% Pedro Revenga, Fernando Herránz and Noelia Hernández. Thanks a lot to
% all of them, and to the many anonymous contributors found (thanks to
% google) that provided help in setting all this up.
% 
% See also the additionalContributors.txt file to check the name of
% additional contributors to this work.
% 
% If you think you can add pieces of relevant/useful examples,
% improvements, please contact us at (macias@depeca.uah.es)
% 
% You can freely use this template and please contribute with
% comments or suggestions!!!
% 
%%%%%%%%%%%%%%%%%%%%%%%%%%%%%%%%%%%%%%%%%%%%%%%%%%%%%%%%%%%%%%%%%%%%%%%%%%% 

\chapter{Introduction}
\label{cha:introduction}

\begin{FraseCelebre}
  \begin{Frase}
    Aaay, el oro, la fama, el poder.  \\
    Todo lo tuvo el hombre que en su día se autoproclamó  \\
    el rey de los piratas, ¡GOLD ROGER!  \\
    Mas sus últimas palabras no fueron muy afortunadas:  \\
    "¿¡MI TESORO!? Lo dejé todo allí, buscadlo si queréis,  \\
    ojalá se le atragante al rufián que lo encuentre.
  \end{Frase}
  \begin{Fuente}
    Opening 1 de One Piece: "We are" \\
    Autor original: Hiroshi Kitadani
  \end{Fuente}
\end{FraseCelebre}

\section{Motivation}
\label{sec:1_motivation}

Autonomous Vehicles (AVs) have held the attention of technology enthusiasts and futurists
for some time as evidenced by the continuous development and research in Autonomous
Vehicle Technologies (AVT) over the past two decades, being one of the emerging
technologies of the Fourth Industrial Revolution, and particularly of the Industry 4.0.
The phrase Fourth Industrial Revolution was first introduced by Klaus Schwab, CEO (Chief
Executive Officer) of the World Economic Forum, in a 2015 article in Foreign Affairs
(American magazine of international relations and United States foreign policy). A
technological revolution is defined as a period in which one or more technologies are
replaced by other kinds of technologies in a short amount of time. Hence, it is an era of
accelerated technological progress featured by Researching, Development and Innovation
whose rapid application and diffusion cause an abrupt change in society. In particular, the
Fourth Industrial Revolution is expected to be marked by breakthroughs in emerging
technologies in fields such as Artificial Intelligence (AI), Computer Vision, Internet of Things
(IoT), fifth-generation wireless technologies (5G), Robotics, 3D printing and the scope of
this master thesis, fully autonomous vehicles. The sum of all these advances are resulting in
machines that can potentially see, hear and what is more important, think, moving more
deftly than humans.
Moreover, Industry 4.0 is the subset of the Fourth Industrial Revolution that concerns
industry, that is, this concept focuses the existence of factories in which machines are
enhanced with sensors and wireless connectivity, connected to a system that can visualize
the whole production line and make decision on its own. In fact, if it is substituted “the whole
production” by the environment, the concept refers to a self-driving car.
A self-driving car (also known as driverless car or autonomous car) is a vehicle that can
sense its environment and moving safely with little or even no human input. They combine
a variety of sensors to recognize their environments, such as GPS, camera, Inertial
Measurements Units (IMUs), radar, sonar or LiDAR (Light Detection and Ranging). Then,
advanced control systems process this sensory information in order to calculate in a proper
way navigation paths, traffic signs or detect and track the road obstacles (which is the main
purpose of this thesis) to ensure a safe driving.
Furthermore, statistics show that 69 % of the population in the European Union (EU),
including associated states, lives in urban areas. According to the World Health
Organization, nearly one third of the world population will live in cities by 2030, leading to
an overpopulation in most of them. Aware of this problem, the Transport White Paper published by the European Commission in 2011 indicated that new forms of mobility ought
to be proposed so as to provide sustainable solutions for people and goods safely. For
example, regarding safety, it sets the ambitious goal of halving the overall number of road
deaths in the EU between 2010 and 2020. Nevertheless, this goal does not seem to be easy
since only in 2014 more than 25,700 people died on the roads in the EU, many of them
caused by an improper behaviour of the driver on the road.
Autonomous driving is considered as one of the solutions to the before mentioned problems
and one of the greatest challenges of the automotive industry today. The existence of
reliable and economically affordable autonomous vehicles will create a huge impact on
society affecting social, demographic, environmental and economic aspects. Besides this, it
is estimated to cause a reduction in road deaths, reduce fuel consumption and harmful
emission associated and improve traffic flow, as well as an improvement in the overall
driver comfort and mobility in groups with impaired faculties, such as disable or elderly
people. Other industrial applications of autonomous vehicles are agriculture, retail,
manufacturing, commercial and freight transport or mining.

\section{Historical Context}
\label{sec:1_historical_context}

\section{Problem Statement}
\label{sec:1_problem_statement}

\cite{huang2022survey}

\section{Objectives and Structure of this work}
\label{sec:1_objectives_and_structure}

Most of the current tracking systems are based on traditional techniques. In that sense, the
main scope of this work is to study the state-of-the-art of Deep Learning based Multi-Object
Tracking approaches and implement and validate an optimal architecture both in
simulation and real world, mainly focused on the Autonomous Vehicles paradigm. It is a
hard issue due to the lack of literature in this specific branch of deep-learning based object
tracking applications, as shown in [2] [3] [4]. Moreover, in order to achieve the main scope,
the following objectives will be met:
1. Researching of current Deep-Learning based Multi-Object Tracking approaches.
2. Study of state-of-the-art software technologies and sensors to perform the MOT
problem both in simulation and real-world.
3. Explanation of a real-world project named SmartElderlyCar, including its hardware and
software architecture.
4. Propose an architecture for Deep-Learning based Multi-Object Tracking.
5. Validate the proposed architecture for MOT both in CARLA simulator and real world

The organization of this document has been done as follows:
• Chapter 2 presents a technical background about current object tracking
approaches, including Visual Object Tracking, LiDAR based and sensor fusion. Then,
as this master thesis focuses of 2D tracking, challenges in Visual Object Tracking are
shown. Then, Deep Learning in MOT is studied in addition to some state-of-the-art
approaches.
• Chapter 3 focuses on the software technologies used in this master thesis, that is,
ROS for sensor communication, PCL as point cloud processing, Docker as a tool to
increase the portability and testability of the project and CARLA as simulator
environment.
• Chapter 4 presents the SmartElderlyCar project, an autonomous electric car able to
drive in the University of Alcalá campus, as the reference to develop the architecture
proposal of this work.
• Chapter 5 shows the Deep Learning based Multi-Object Tracking architecture
proposal. This is the main chapter of this master thesis.
• Chapter 6 shows the validation of the architecture proposal in the KITTI benchmark
as well as quantitative and qualitative results both in CARLA simulator, KITTI
benchmark and in the real prototype of the SmartElderlyCar.
• Chapter 7 illustrates the conclusions and future works of this project.
• Appendix A details how Kalman filter works in the object tracking context.
• Appendix B shows the Artificial Intelligence paradigm, including Machine Learning
and Deep Learning concepts. In addition, a brief explanation of how CNNs and RNNs
work due to its tight relation with object detection and tracking.
• Appendix C shows some interesting parts of the code created and developed in order
to perform most of exposed tasks throughout this master thesis.
• Appendix D illustrates the user’s manual so as to install the system requirements and
reproduce the obtained results.
• Appendix E represents the main hardware and software specifications used in this
project.
• Appendix F illustrated an estimation of the required budget to develop this thesis.
