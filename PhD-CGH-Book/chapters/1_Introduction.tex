%%%%%%%%%%%%%%%%%%%%%%%%%%%%%%%%%%%%%%%%%%%%%%%%%%%%%%%%%%%%%%%%%%%%%%%%%%% 
% 
% Generic template for TFC/TFM/TFG/Tesis
% 
% By:
% + Javier Macías-Guarasa. 
% Departamento de Electrónica
% Universidad de Alcalá
% + Roberto Barra-Chicote. 
% Departamento de Ingeniería Electrónica
% Universidad Politécnica de Madrid   
% 
% Based on original sources by Roberto Barra, Manuel Ocaña, Jesús Nuevo,
% Pedro Revenga, Fernando Herránz and Noelia Hernández. Thanks a lot to
% all of them, and to the many anonymous contributors found (thanks to
% google) that provided help in setting all this up.
% 
% See also the additionalContributors.txt file to check the name of
% additional contributors to this work.
% 
% If you think you can add pieces of relevant/useful examples,
% improvements, please contact us at (macias@depeca.uah.es)
% 
% You can freely use this template and please contribute with
% comments or suggestions!!!
% 
%%%%%%%%%%%%%%%%%%%%%%%%%%%%%%%%%%%%%%%%%%%%%%%%%%%%%%%%%%%%%%%%%%%%%%%%%%% 

\chapter{Introduction}
\label{cha:introduction}

\begin{FraseCelebre}
  \begin{Frase}
    Aaay, el oro, la fama, el poder.  \\
    Todo lo tuvo el hombre que en su día se autoproclamó  \\
    el rey de los piratas, ¡GOLD ROGER!  \\
    Mas sus últimas palabras no fueron muy afortunadas:  \\
    "¿¡MI TESORO!? Lo dejé todo allí, buscadlo si queréis,  \\
    ¡Ojalá se le atragante al rufián que lo encuentre!.
  \end{Frase}
  \begin{Fuente}
    Opening 1 de One Piece: "We are" \\
    Autor original: Hiroshi Kitadani
  \end{Fuente}
\end{FraseCelebre}

\section{Motivation}
\label{sec:1_motivation}

\ac{AD} have held the attention of technology enthusiasts and futurists for some time as is evidenced by the continuous research and development of this topic in \ac{ITS} over the past decades, being one of the emerging technologies of the \textit{Fourth Industrial Revolution}, and particularly of the Industry 4.0. 

The concept \textit{Fourth Industrial Revolution} or Industry 4.0  was first introduced by Klaus Schwab, CEO (Chief Executive Officer) of the World Economic Forum, in a 2015 article in Foreign Affairs (American magazine of international relations and United States foreign policy). A technological revolution can be defined as a period in which one or more technologies are replaced by other kinds of technologies in a short amount of time. Hence, it is an era of accelerated technological progress featured by Researching, Development and Innovation whose rapid application and diffusion cause an abrupt change in society. In particular, the \textit{Fourth Industrial Revolution} conceptualizes rapid change to industries, technology, processes and societal patterns in the 21st century due to increasing inter-connectivity and smart automation. This industrial revolution focuses on operational efficiency, being the following four themes which summarize it: 

\begin{itemize}
	
	\item \textbf{Decentralized decisions}: Ability of cyber physical systems to make decisions on their own and to perform their tasks as autonomously as possible.
	
	\item \textbf{Information transparency}: Provide operators with comprehensive information to make decisions. Inter-connectivity allows operators to gather large amounts of information and data from all points in the manufacturing process in order to identify key areas or aspects that can benefit from improvement to enhance functionality.
	
	\item \textbf{Technical assistance}: Ability to assist humans with unsafe or difficult tasks and technological facility of systems to help humans in problem-solving and \ac{DM}.
	
	\item \textbf{Interconnection}: Ability of machines, sensors, devices and people to communicate and conect with each other via the Internet of Things (IoT) or the Internet of People (IoP).
	
\end{itemize}

Based on the aforementioned principles, this revolution is expected to be marked by breakthroughs in emerging technologies in fields such as nanotechnology, quantum computing, 3D printing, Internet of Things (IoT), fifth-generation wireless technologies (5G), Robotics, Computer Vision (CV), \ac{AI} or the scope of this doctoral thesis, \acp{ADS}. The sum of all these advances are resulting in machines that can potentially see, hear and what is more important, think, moving more deftly than humans. 

An \ac{ADS}, driverless car or autonomous car, is a vehicle than can sense its surrounding and moving safely with little or even no human intervention. These \acp{ADS} must combine a variety of sensors to understand the traffic scenario, like \ac{RADAR}, \ac{LiDAR}, cameras, Inertial Measurement Unit (IMU), wheel odometry, GNSS (Global Navigation Satellite System) or ultrasonic sensors, in order to detect, track and predict (which is the main purpose of this thesis) the most relevant obstacles around the ego-vehicle. Then, advanced control and planning systems process this sensory information in combination with a predefined global route to calculate the corresponding control commands to drive the vehicle throughout the environment, ensuring a safe driving. 

The dream of seeing fleets of \acp{ADS} efficiently delivering goods and people to their destination has fueled billions of dollars and captured consumer's imaginations in investment in recent years. Nevertheless, according to the \textit{Autonomous driving's future: Convenient and connected} report, published by the global management consulting firm McKinsey \& Company in January 2023, even after some setbacks have pushed out timelines for \ac{AD} launches and delayed customer adoption, the transportation community still broadly agrees that \ac{AD} has the potential to transform consumer behaviour, transportation and society at large. \ac{AD} is considered as one of the solutions to the aforementioned problems and one of the greatest challenges of the automotive industry today. 

The World Health Organization (WHO) has indicated that the global population is increasingly concentrated in cities, which has significant implications for public health. The trend of urbanization has been observed for several decades, with people migrating from rural to urban centers in search of better opportunities and improved living conditions. As a result, cities are becoming more crowded, and the WHO predicts that by the year 2050, nearly 70 \% of the world population will reside in urban areas.

This concentration of people in cities poses both challenges and opportunities for public health. On the positive side, cities offer access to better healthcare facilities, educational institutions, and employment opportunities. Urban areas also tend to have improved sanitation systems and infrastructure, which can contribute to better overall health outcomes. However, the rapid growth of cities can strain existing resources and lead to overcrowding, inadequate housing, and increased pollution levels, all of which can have adverse effects on public health.

Aware of this problem, the European Commision (EU) has set several objectives and statistics to improve transport mobility in the future. These objectives and statistics are part of the EU broader vision for a sustainable, efficient, and interconnected transport system to create a seamless, sustainable, and integrated transport system across member states:

\begin{itemize}
	
	\item \textbf{Sustainable Transport}: The European Commission aims to promote sustainable modes of transport, such as railways, public transport, cycling, and walking, to reduce greenhouse gas emissions, congestion, and air pollution. The target is to achieve a 60 \% reduction in transport emissions by 2050 compared to 1990 levels.
	
	\item \textbf{Infrastructure Investment}: The EU has committed to investing in modern and efficient transport infrastructure, including the Trans-European Transport Network (TEN-T) and the Connecting Europe Facility (CEF). These investments aim to improve connectivity, remove bottlenecks, and enhance multi-modal transport options across the EU.
	
	\item \textbf{Road Safety}: The European Commission has prioritized improving road safety to reduce the number of accidents, injuries, and fatalities. The objective was to reduce road fatalities by 50 \% by 2030 compared to 2020 levels. This includes implementing measures such as stricter vehicle safety standards, promoting safe driving behaviors, and improving road infrastructure.
	
	\item \textbf{Digitalization and Innovation}: The EU aims to harness digital technologies and innovation to improve transport efficiency, reliability, and user experience. This includes initiatives such as \acp{ITS}, the use of big data for planning and optimization, and the development of Connected and Autonomous Vehicles (CAVs) to enhance mobility.
	
	\item \textbf{Freight Transport Efficiency}: The European Commission aims to improve the efficiency and sustainability of freight transport through measures such as promoting inter-modal transport, increasing the use of clean and energy-efficient vehicles, and implementing logistics optimization strategies. The objective is to reduce external costs, including congestion, noise, and emissions, associated with freight transport.
	
\end{itemize}
 
In that sense, the existence of reliable and economically affordable \acp{ADS} are expected to create a huge impact on society affecting social, demographic, environmental and economic aspects. It can produce substantial value for the automotive industry, drivers and society, making driving safer, more convenient and more enjoyable. In other words, the hours on the road previously spent by manual driving could be used to work, watch a funny movie or even to video call a friend. For employees with long commutes, \ac{AD} might shorten the workday, increasing worker productivity. Since workers, specially those related to digital jobs or related fields, may perform their jobs from an \ac{ADS}, they could more easily move further away from the office, which, in turn, could attract more people to suburbs and rural areas. Besides this, it is estimated to cause a reduction in road deaths, reduce fuel consumption and harmful emission associated and improve traffic flow, as well as an improvement in the overall driver comfort and mobility in groups with impaired faculties, such as disable or elderly people, providing them with mobility options that go beyond car-sharing services or public transportation. Other industrial applications of autonomous vehicles are agriculture, retail, manufacturing, commercial and freight transport or mining. 

\section{Historical Context}
\label{sec:1_historical_context}

\acp{ADS} have become a challenge for automation and technology companies, which has derived in an intense competition. Though today companies such as Mercedes, Ford or Tesla are racing to build \acp{ADS} for a radically changing consumer world, the research and development of autonomous robots is not new.

In 1500, centuries before the invention of the automobile, Leonardo da Vinci designed a cart that could move without being pulled or pushed. In 1868, Robert Whitehead invented a torpedo that could propel itself underwater in order to be a game-changer for naval fleets all over the world. In terms of robotic solutions for intelligent mobility, the study was started in the 1920s, being the concept of Autonomous Car defined in Futurama, an exhibit at the 1939 New York Wolrd's Fair. General Motors created the exhibit to display its vision of what the world would look like in 20 years, including an automated highway system that would guide \ac{ADS}. By 1958, General Motors made this concept a reality (at least as a proof of concept) being the car's front end embedded with sensors to detect the current flowing through a wire embedded in the road. The first semi-automated car was developed in 1977 by Japan’s Tsukuba Mechanical Engineering Laboratory. The vehicle reached speeds up to 30 km/h with the support of an elevated rail. 

\begin{figure}[h]
	\centering
	\includegraphics[width=0.8\linewidth]{chapter_1_intro/darpa_winner_2005.png}
	\caption[Stanley, 2005 DARPA Grand Challenge winner]{Stanley, 2005 DARPA Grand Challenge winner \\ Source: \textit{Stanford university}}
	\label{fig:chapter_1_intro/darpa_winner_2005}
\end{figure}

Nevertheless, the first truly autonomous cars appeared in the 1980s with Carnegie Mellon University’s Navlab and ALV projects funded by the USA company DARPA (Defense Advanced Research Projects Agency) in 1984 and EUREKA Prometheus project (1987) developed by Mercedes-Benz and Bundeswehr University Munich’s. By 1985, the ALV project had shown self-driving speeds on two-lane roads of 31 km/h with obstacle avoidance added in 1986 and off-road driving in day and night conditions by 1987. Furthermore, from the 1960s through the second DARPA Grand Challenge in 2005 (212 km off-road course near the California-Nevada state line, surpassed by all but one of the 23 finalists), automated vehicle research in the United States was primarily funded by DARPA, the US Army and US Navy, yielding rapid advances in terms of speed, car control, sensor systems and driving competence in more complex conditions. This caused a boost in the development of autonomous prototypes by companies and research organizations, most of them from the United States. Figure \ref{fig:chapter_1_intro/darpa_winner_2005} shows Stanley, the 2005 DARPA Gran Challenge winner, from Stanford university. 

Even though self-driving cars have not yet displaced conventional cars, there can be found several examples of how it has become a hot topic for powerful companies such as Delphi Automotive Systems, Audi, BMW, Tesla, Mercedes-Benz or Waymo. In 2005 Delphi broke the Navlab’s record achievement (driving 4,584 km while remaining 98 \% of the time autonomously) by piloting an Audi, improved with Delphi technology, over 5,472 km through 15 states while remaining in self-driving mode 99 \% of the time. Moreover, in 2005 the USA states of Michigan, Virginia, California, Florida, Nevada and the capital, Washington D.C., allowed the testing of automated cars on public roads. 

In 2017, Audi stated that its A8 car prototype would be automated at speeds up to 60 km/h by using its perception system named “Audi AI”.  Also, in 2017 Waymo (self-driving technology development company subsidiary of Alphabet Inc) started a limited trial of a self-driving taxi service in Phoenix, Arizona. 

Figure \ref{fig:chapter_1_intro/disengagement_2020} shows the total number of autonomous test miles and miles per disengagement in California (Dec 2019 - Nov 2020) by some of the most important \ac{AD} technology development companies around the world. The concept disengagement is quite useful to assess the quality of an \ac{ADS}, defined as the deactivation of the autonomous mode when a failure of the autonomous technology is detected or when a safe operation requires that the autonomous vehicle test driver disengages the autonomous mode, resulting in control being seized by the human driver. As observed, Waymo and Cruise were, by far, the companies with the highest number of miles per disengagement, indicating the superiority and stability of their \ac{ADS}. Furthermore, we can conclude from Figure \ref{fig:chapter_1_intro/disengagement_2020} that most \ac{AD} companies are from the United States of America (specially from the state of California) and China, being United States clearly the most developed country in this field.  

\begin{figure}[ht]
	\centering
	\includegraphics[width=0.6\linewidth]{chapter_1_intro/disengagement_2020.png}
	\caption{Number of autonomous test miles and miles per disengagement (Dec 2019 - Nov 2020)}
	Source: \textit{DMV California, via The Last Driver License Holder}
	\label{fig:chapter_1_intro/disengagement_2020}
\end{figure}

At the moment of writing this thesis (2023), most vehicles on the road are considered to be semi-autonomous due to presence of \acp{ADAS} that include assisted parking, lane departure warning, driver monitoring, emergency break or \ac{ACC}, among others. Regarding this, the Society of Automotive Engineers (SAE) published the concept of autonomy levels in 2014, as part of its \textit{Taxonomy and Definitions for Terms Related to On-Road Motor Vehicle Automated Driving Systems} \cite{taxonomy2016definitions} report. Figure \ref{fig:chapter_1_intro/nhtsa_sae_automation_levels} illustrates the six levels of autonomy (the higher the level, the more autonomous the car is), where it can be appreciated that Level Zero means \textit{No Automation}, being the acceleration, braking and steering controlled by a human driver at all times, and Level Five represents Full Automation, where there is a full-time automation of all driving tasks on any road, under any conditions, whether there is a human on board or not.

\begin{figure}[h]
	\centering
	\includegraphics[width=\linewidth]{chapter_1_intro/nhtsa_sae_automation_levels.png}
	\caption{Society of Automotive Engineers (SAE) automation levels}
    Source: \textit{NHTSA (National Highway Traffic Safety Administration)}
	\label{fig:chapter_1_intro/nhtsa_sae_automation_levels}
\end{figure}

% In that sense, today most vehicles only included basic Advanced Driver Assistance Systems (ADAS), but major advancements in AD capabilities are on the horizon. 

According to a 2021 McKinsey consumer survey, growing demand for \ac{AD} systems could create billions of dollars in revenue. Based on a consumer interest in \ac{AD} features and commercial solutions available on the market today, ADAS and AD could generate between \$300 and \$400 billions in the passenger car market by 2035. Figure \ref{fig:chapter_1_intro/mckinsey_revenues_ad} illustrates an interesting study reporting the revenues of \ac{ADAS} and \ac{AD}  from Level 1 (Driver Assistance) to Level 4 (High Automation). As expected, Level 5 is excluded from this study due to the huge difficulties the automotive companies would have to face to adapt their systems under totally different environmental conditions.

\begin{figure}[h]
	\centering
	\includegraphics[width=0.8\linewidth]{chapter_1_intro/mckinsey_revenues_ad.png}
	\caption{Advanced Driver Assistance systems (ADAS) and \\ Autonomous Driving (AD) revenues in \$ billion} Source: \textit{McKinsey Center for Future Mobility}
	\label{fig:chapter_1_intro/mckinsey_revenues_ad}
\end{figure}

So far this Chapter has focused on a commercial analysis of the \ac{ITS} field. The remaining content of this Chapter focuses on the technical study of an \ac{ADS}, problem statement and objectives of this thesis.

\section{Autonomous Driving Stack}
\label{sec:1_ad_architecture}

To sum up what commented above, increasing the level of autonomous navigation in mobile robots (from agriculture to public and private transport) are expected to create tangible business benefits to those users and companies employing them. However, designing an \ac{ADS} does not seem to be an easy task. In the \ac{SOTA} we can distinguish two main kind of software architectures: End-to-End and modular. Note that in this thesis we focus on the software components of the \ac{ADS}, not on the hardware tasks of the vehicle. 

Figure \ref{fig:chapter_1_intro/ete_modular} illustrates the entire \ac{AD} architecture starting from sensing to longitudinal (throttle/brake) and lateral (steering angle) control of the vehicle, which are the commanded signals that feed the low-level electronic system that moves the vehicle and that is known as the Drive-By-Wire system \cite{arango2020drive}. End-to-End are considered black-box models, where a single neural network performs the whole driving task (throttle/steering/brake) from raw sensor data, in such a way the error be may vanished since intermediate representations are jointly optimized, but these are not very interpretable. On the other hand, modular architectures (considered as glass models as counterpart to End-to-End approaches) separate the driving task into individually programmed or trained modules. This solution is more interpretable, since the know-how of a research group or company is easily transferred, they allow parallel development, being the standard solution in industrial research, but the error is propagated, where intermediate representations can led to sub-optimal performance. For example, incorrect object detection can lead to low-quality tracking and \ac{MP}.

\begin{figure}[h]
	\centering
	\includegraphics[width=0.8\linewidth]{chapter_1_intro/ete_modular.png}
	\caption{Autonomous Driving Stack (ADS) modular vs end-to-end pipeline}
	Source: \textit{Vrunet: Multi-task learning model for intent prediction of vulnerable road users} \cite{ranga2020vrunet}
	\label{fig:chapter_1_intro/ete_modular}
\end{figure}

Considering the RobeSafe (Robotics and eSafety) research group and the main projects (Techs4AgeCar, AIVATAR) where this thesis has been developed, we integrate our algorithms in a software modular approach. An example of this modular approach is shown in Figure \ref{fig:chapter_1_intro/pylot_architecture}. Despite the fact in literature some authors disagree on the specific software architecture of an \ac{ADS}, specially the motion prediction module, which is usually classified as a perception algorithm but sometimes is included as part of the planning or \ac{DM} layers, we can hierarchically break down (from raw data to the driving task) a standard \ac{AD} architecture into the following software layers:

\begin{itemize}
	
	\item \textbf{Localization layer}: Positions the vehicle on a map with real-time and centimetric accuracy approach. The main source of information is a robust differential-GNSS, though IMUs, wheel odometry and even cameras.
	
	\item \textbf{Perception layer}: Understands the environment around the ego-vehicle thanks to the information collected by the sensors. If defined as multi-stage, the perception layer first detects the most relevant obstacles, then tracks them over time and predicts their trajectories. In that sense, the perception layer represents one of the most important modules of an \ac{ADS}, responsible of analyzing the online information, also referred as the traffic situation, through the use of a global perception system which involves different on-board sensors as: \ac{LiDAR}, Inertial Measurement Unit (IMU), \ac{RADAR}, Differential-Global Navigation Satellite System (D-GNSS), Wheel odometers or Cameras. Additionally, \ac{HDmap} information is frequently used in the \ac{MP} tasks by most \ac{SOTA} algorithms.

	\item \textbf{Mapping layer}: Responsible for creating a topological, semantic and geographical modeling of the environment through which the vehicle drives, being the \ac{HDmap}p graph the most common source of information.
	
	\item \textbf{Planning layer}: This layer is comprised of three components: route, behaviour and trajectory planner. The route planner computes the most optimal (in terms of distance, time and so forth and so on) global route from some predefined start and goal. It uses the localization and mapping output. On the other hand, the behaviour planner, also referred as \ac{DM} layer by some authors, performs high-level \ac{DM} of driving behaviours such as lane changes or progress through intersections, mostly focused on the previously computed global route and current localization. It can be seen as an atomization of the global route in different behaviors to reach the goal. Finally, the trajectory planner, also known as local planner, generates a time schedule for how to follow a path given constraints such as position, velocity and acceleration in order to meet the previously decided behaviour and taking into account the prediction from the perception layer, avoiding obstacles in optimal direction and speed conditions.
	
	\item \textbf{Control layer}: Once the local plan is calculated, the control layer is responsible for generating the commands that are sent to the actuators. It receives as input some waypoints from the trajectory planner and most authors perform spline interpolations and a velocity profile that ensures a smooth and continous trajectory. 
	
\end{itemize}

\begin{figure}[h]
	\centering
	\includegraphics[width=0.7\linewidth]{chapter_1_intro/pylot_architecture.png}
	\caption{Autonomous Driving Stack (ADS) modular pipeline}
	Source: \textit{Pylot: A modular platform for exploring latency-accuracy trade-offs in autonomous vehicles} \cite{gog2021pylot}
	\label{fig:chapter_1_intro/pylot_architecture}
\end{figure}

\section{Problem statement}
\label{sec:1_problem_statement}

As commented in previous sections, in order to operate efficiently and safely in highly dynamic, complex and interactive driving scenarios, \acs{ADS} need to smartly reason like human beings via predicting future motions of surrounding traffic participants during navigation. Nevertheless, achieving accurate and robust \ac{MP} is one of the most difficult challenges to achieve full-autonomy, since it is equivalent to a bridge between the former stages of the perception layer, where the scene is understood detecting and tracking static and dynamic objects of the environment, and the planning and control layer, where the future trajectory of the ego-vehicle is computed and the driving commands are sent to the physical layer (e.g. Drive-by-Wire \cite{arango2020drive}). Here are some of the most important challenges: 

\begin{itemize}
	
	\item \textbf{Heterogeneity of traffic participants}: Traffic participants (specially those which are dynamic) can be roughly classified as cyclists, pedestrians or other vehicles. The prediction model should be capable of differentiating the motion patterns of heterogeneous traffic participants, in such a way fine-grained classification (detection module) is quite beneficial to include additional metadata along with the past observations.
	
	\item \textbf{Complexity of road structure}: Road structures are highly diverse and complex, specially in highways and urban areas, which noticeably affect the motion behaviours of traffic participants.
	
	\item \textbf{Variable number of interactive agents}: The prediction model must deal with a number of associated traffic participants within a certain area that can vary from time to time, such as intersections or roundabouts. Then, while driving, a comprehensive representation of the scene must be able to accommodate an arbitrary number of involved traffic participants.
	
	\item \textbf{Multi-modality of driving behaviours}: In real-world, despite we know the behaviour our vehicle will carry out, the motion patterns of other traffic participants can be considered inherently multi-modal since there is usually more than one reasonable option for a driver to choose, specially in intersections, when the number of lanes increases or even in the same lane with different velocity profiles (constant velocity, sudden break, sudden acceleration). In that sense, a robust and reliable \ac{MP} model is expected to be human-like and capture different plausible motion modalities where an agent can travel in the prediction horizon.
	
	\item \textbf{Complex interpendencies among traffic participants and road infrastructure}: Agent-Agent, Agent-Road and Road-Road interpendencies are of great importance for \ac{MP} and interaction modeling, even more taking into account the complexity of road structures and heterogeneity of traffic participants aforementioned. As expected, an agent future trajectory will be affected not only by its own past trajectory and driving objectives (given by the behaviour planner) but also by other surrounding agents past trajectories, traffic rules and physical constraints.
	
\end{itemize}

\section{Contributions of this Thesis}
\label{sec:1_objectives_and_contributions}

The main scope of this thesis is to develop novel and efficient interaction-aware \ac{DL} based \ac{MP} models in the field of \ac{AD}, focusing on long-term (from 3 to 6 s) prediction horizon and \ac{AD}, where traffic participants can range from trucks to pedestrians. The main inputs will be the physical (map) information and historical states (that may include agent position, velocity, orientation, object type and category) of traffic participants in \ac{BEV}, assuming these objects have been previously tracked by our ego-vehicle (also referred as the autonomous car). 

The validation of these methods will be done using a single target agent (either the model has considered multiple or a single agent in the loss function), as proposed by some of the most important prediction datasets in the literature, like Argoverse 1 \cite{chang2019argoverse} and Argoverse 2 \cite{wilson2023argoverse}, at the moment of writing this thesis. In this work, the solutions to the aforementioned challenges will be discussed and investigated progressively. 

We summarise the contributions of this doctoral thesis in the following points:

\begin{itemize}
	
	\item \textbf{Contribution 1}: SmartMOT, a simple-yet-accurate combination of traditional techniques is proposed for state estimation and data association respectively, in order to solve the tracking-by-detection paradigm. Moreover, we incorporate \ac{HDmap} semantic, geometric and topological information, in addition to the ego-vehicle status, to enhance the efficiency and reliability of the \ac{MOT} system and subsequent uni-modal predictions (Chapter \ref{cha:smartmot_exploiting_the_fusion_of_hdmaps_and_mot}).
	
	\item \textbf{Contribution 2}: An Attention-based \ac{GAN} prediction model that can successfully predict plausible uni-modal future trajectories in the context of vehicle prediction (Argoverse 1 dataset), taking into account not only the past trajectory of the agents (encoded by \ac{LSTM} networks and attention mechanisms) but also the \ac{HDmap} information to compute a set of acceptable target points representing the physical constraints for our problem (Chapter \ref{cha:exploring_gan_for_vehicle_mp}).
	
	\item \textbf{Contribution 3}: Three efficient (social, map and augmented) baselines for multi-modal \ac{MP} in the Argoverse 1 dataset built upon the previous model proposed in Chapter \ref{cha:exploring_gan_for_vehicle_mp} removing the adversarial framework. In this case, we integrate \acp{GNN} to compute complex interactions among the different agents and a heuristic-based method to calculate some preliminary future centerlines for the vehicles as an enhanced interpretable representation of the map information with respect to the acceptable target points aforementioned (Chapter \ref{subsec:6_efficient_baselines_social}, \ref{subsec:6_efficient_baselines_map_baseline} and \ref{subsec:6_augmented_baseline}). 
	
	\item \textbf{Contribution 4}: A multi-modal multi-agent \ac{MP} model in the Argoverse 2 dataset, built upon the augmented efficient baseline, which incorporates topological and semantic information of preliminary future lanes using the aforementioned heuristic method, map encoding based on \ac{DL} with \ac{GNN}, a cycle of physical and social feature fusion, \ac{DL}-based estimation of final positions on the road, aggregation of the surrounding environment, and finally, a refinement module to enhance the quality of the final multi-modal predictions in an elegant and efficient manner. Compared to the state of the art, our method achieves prediction metrics up-to-pair with to the top-performing methods on the Argoverse 2 Leaderboard while significantly reducing the number of parameters and floating-point operations per second (Chapter \ref{cha:improving_multi_agent}).
	
	\item \textbf{Contribution 5}: The final model of the thesis, only considering social information, is validated in two interesting applications, such as \ac{DM} in the SMARTS simulator or domain adaptation studies in the hyper-realistic CARLA simulator, involving other vehicle layers as a preliminary step towards implementation in a real autonomous vehicle (Chapter \ref{sec:8_decision_making} and \ref{sec:8_domain_adaptation_carla}). 
	 
\end{itemize}

\begin{comment}
In order to achieve the main scope, the following objectives will be met:

\begin{enumerate}
	
	\item Review of \ac{SOTA} \ac{MP}, focused on \ac{DL} and the \ac{AD} paradigm.
	
	\item Propose of several efficient \ac{MP} architectures, studying the progressive incorporation of \ac{DL} mechanisms and different sources of information and metadata, achieving \ac{SOTA} accuracy while reducing in millions of parameters previous models as well as inference time.
	
	\item Validate the proposed models in downstream applications, such as \ac{DM} or behaviour planning, taking into account former stages of the perception layer (detection and tracking) instead of static files (benchmarks) in hyper-realistic simulation, as a preliminary stage before implementing it in a real-world vehicle.
	
\end{enumerate}
\end{comment}

\section{Structure of this Thesis}
\label{sec:1_structure}

The organization of this document has been done as follows:

\begin{itemize}
	
	\item \textbf{Chapter 2} reviews the contextual factors, a \ac{MP} methods classification according to the context encoding or representation approaches and \ac{SOTA} databases and simulators to validate the algorithms. 
	
	\item \textbf{Chapter 3} presents a technical background, mostly focused on physics-based methods and \ac{DL} mechanisms to deal with temporal sequences and interactions, to deeply understand the proposed methods.
	
	\item \textbf{Chapter 4} addresses our integration between single-yet-powerful \ac{MOT} and \ac{HDmap} information as a preliminary stage before computing uni-modal predictions.
	
	\item \textbf{Chapter 5} illustrates our \ac{GAN}-based proposal, the first \ac{DL}-based vehicle \ac{MP} method of this thesis, considering both physical context, computing acceptable target points from the driveable area around the target agent, and social context, computing the past trajectory as a temporal sequence via recurrent networks and social interactions with attention mechanism.
	
	\item \textbf{Chapter 6} presents our efficient baselines, where high-level and well-structured physical context is structured in the form of centerlines, \ac{GCN}-based approaches are studied to model more complex agent-agent interactions and transformer encoders are employed at the end of the chapter for powerful-yet-efficient context encoding and multi-modal decoding.
	
	\item \textbf{Chapter 7} illustrates the final model of the thesis which takes into account agent-agent, agent-map, map-agent and map-map interactions, using a novel scene representation with heuristic proposals, graph-based encoding, \ac{DL}-based goal areas proposals and motion refinement.
	
	\item \textbf{Chapter 8} addresses the integration and validation of the final model of the thesis in a hyper-realistic simulator with upstream and downstream modules to contribute the entire pipeline and closed-loop for \ac{AD}.
	
	\item \textbf{Chapter 9} summarizes the thesis and provides some promising directions for future work in the areas of \ac{MP} and validation.
	
\end{itemize}
