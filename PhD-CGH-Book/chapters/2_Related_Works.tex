%%%%%%%%%%%%%%%%%%%%%%%%%%%%%%%%%%%%%%%%%%%%%%%%%%%%%%%%%%%%%%%%%%%%%%%%%%% 
% 
% Generic template for TFC/TFM/TFG/Tesis
% 
% By:
% + Javier Macías-Guarasa. 
% Departamento de Electrónica
% Universidad de Alcalá
% + Roberto Barra-Chicote. 
% Departamento de Ingeniería Electrónica
% Universidad Politécnica de Madrid   
% 
% Based on original sources by Roberto Barra, Manuel Ocaña, Jesús Nuevo,
% Pedro Revenga, Fernando Herránz and Noelia Hernández. Thanks a lot to
% all of them, and to the many anonymous contributors found (thanks to
% google) that provided help in setting all this up.
% 
% See also the additionalContributors.txt file to check the name of
% additional contributors to this work.
% 
% If you think you can add pieces of relevant/useful examples,
% improvements, please contact us at (macias@depeca.uah.es)
% 
% You can freely use this template and please contribute with
% comments or suggestions!!!
% 
%%%%%%%%%%%%%%%%%%%%%%%%%%%%%%%%%%%%%%%%%%%%%%%%%%%%%%%%%%%%%%%%%%%%%%%%%%% 

\chapter{Related Works}
\label{cha:related_works}

\begin{FraseCelebre}
	\begin{Frase}
		Llegaré a ser el mejor, El mejor que habrá jamás \\
		Mi causa es ser su entrenador, Tras poderlos capturar.  

		Viajaré a cualquier lugar, Llegaré a cualquier rincón \\  
		Y al fin podré desentrañar, El poder de su interior.  

		¡Pokémon! Hazte con todos (solos tú y yo), \\
		Es mi destino, mi misión \\
		¡Pokémon! Tú eres mi amigo fiel, \\
		Nos debemos defender.
	\end{Frase}
	\begin{Fuente}
		Opening 1 de Pokémon: "Gotta catch 'em all!" \\
		Autor original: Jason Paige
	\end{Fuente}
\end{FraseCelebre}

\section{Introduction}
\label{sec:2_introduction}


https://www2.eecs.berkeley.edu/Pubs/TechRpts/2020/EECS-2020-111.pdf

\cite{huang2022survey}

\section{Physic-based Motion Prediction}
\label{sec:2_physic_based_mp}

Physics-based trajectory prediction (PTP) methods represent the motion of vehicles with dynamic or kinematic motion models compliant with the laws of physics. A future trajectory is predicted via state evolution based on these models. Linear models (constant velocity (CV) and constant acceleration (CA) models), which assume straight motion, are the simplest models. The straight motion assumption of
CV and CA oversimplifies vehicular motion because of rotation ignorance. Curvilinear models take into consideration the turn rate (TR) of the vehicle. Similar to
linear models, curvilinear models contain two types, namely Constant Turn Rate
and Velocity (CTRV) and Constant Turn Rate and Acceleration (CTRA) models. Coupling steering angle and velocity leads to Constant Steering Angle and
Velocity (CSAV) and Constant t Steering Angle and Acceleration (CSAA) models.
Authors of [25] compare and evaluate these models. More details of these models
can be found in their work. These models can be used for physics-based trajectory
prediction in many ways.
Inspired by the taxonomies provided in [23, 24], we subdivide physics-based methods into state retention-based methods, reachable set-based methods, Monte Carlo
simulation-based methods, and state estimation-based methods. State retention
methods produce a single trajectory according to the target vehicle’s current states,
while reachable set-based methods produce a union of all reachable states. Monte
Carlo simulation-based methods try to approximate the distribution of the target
vehicle’s future motion, and state estimation-based methods introduce uncertainty
estimation into physics-based methods

\section{Deep Learning based Motion Prediction}
\label{sec:2_dl_based_mp}

\section{Vehicle Motion Prediction}
\label{sec:2_vehicle_based_mp}