%%%%%%%%%%%%%%%%%%%%%%%%%%%%%%%%%%%%%%%%%%%%%%%%%%%%%%%%%%%%%%%%%%%%%%%%%%%
%
% Generic template for TFC/TFM/TFG/Tesis
%
% $Id: resumen.tex,v 1.9 2015/06/05 00:10:31 macias Exp $
%
% By:
%  + Javier Macías-Guarasa. 
%    Departamento de Electrónica
%    Universidad de Alcalá
%  + Roberto Barra-Chicote. 
%    Departamento de Ingeniería Electrónica
%    Universidad Politécnica de Madrid   
% 
% Based on original sources by Roberto Barra, Manuel Ocaña, Jesús Nuevo,
% Pedro Revenga, Fernando Herránz and Noelia Hernández. Thanks a lot to
% all of them, and to the many anonymous contributors found (thanks to
% google) that provided help in setting all this up.
%
% See also the additionalContributors.txt file to check the name of
% additional contributors to this work.
%
% If you think you can add pieces of relevant/useful examples,
% improvements, please contact us at (macias@depeca.uah.es)
%
% You can freely use this template and please contribute with
% comments or suggestions!!!
%
%%%%%%%%%%%%%%%%%%%%%%%%%%%%%%%%%%%%%%%%%%%%%%%%%%%%%%%%%%%%%%%%%%%%%%%%%%%

\chapter*{Resumen}
\label{cha:resumen}
\markboth{Resumen}{Resumen}

\addcontentsline{toc}{chapter}{Resumen}

La conducción autónoma es considerada uno de los más grandes retos tecnológicos de la actualidad. Cuando los coches autónomos conquisten nuestras carreteras, los accidentes se reducirán notablemente, hasta casi desaparecer, ya que la tecnología estará testada y no incumplirá las normas de conducción, entre otros beneficios sociales y económicos.

Uno de los aspectos más críticos a la hora de desarrollar un vehículo autónomo es percibir y entender la escena que le rodea. Esta tarea debe ser tan precisa y eficiente como sea posible para posteriormente predecir el futuro de esta misma y ayudar a la toma de decisiones. De esta forma, las acciones tomadas por el vehículo garantizarán tanto la seguridad del vehículo en sí mismo y sus ocupantes, como la de los obstáculos circundantes, tales como viandantes, otros vehículos o infraestructura de la carretera.

En ese sentido, esta tesis doctoral se centra en el estudio y desarrollo de distintas técnicas predictivas para el entendimiento de la escena en el contexto de la conducción autónoma. Durante la tesis, se observa una incorporación progresiva de técnicas de aprendizaje profundo en los distintos algoritmos propuestos para mejorar el razonamiento sobre qué está ocurriendo en el escenario de tráfico, así como para modelar las complejas interacciones entre la información social (distintos participantes o agentes del escenario, tales como vehículos, ciclistas o peatones) y física (es decir, la información geométrica, semántica y topológica del mapa de alta definición) presente en la escena. 

La capa de percepción de un vehículo autónomo se divide modularmente en tres etapas: Detección, Seguimiento (\textit{Tracking}), y Predicción. Para iniciar el estudio de las etapas de seguimiento y predicción, se propone un algoritmo de \textit{Multi-Object Tracking} basado en técnicas clásicas de estimación de movimiento y asociación validado en el dataset KITTI, el cual obtiene métricas del estado del arte. Por otra parte, se propone el uso de un filtro inteligente basado en información contextual de mapa, cuyo objetivo es monitorizar los agentes más relevantes de la escena en el tiempo, representando estos agentes filtrados la entrada preliminar para realizar predicciones unimodales basadas en un modelo cinemático. Para validar esta propuesta de filtro inteligente se usa CARLA (\textit{CAR Learning to Act}), uno de los simuladores hiperrealistas para conducción autónoma más prometedores en la actualidad, comprobando cómo al usar información contextual de mapa se puede reducir notablemente el tiempo de inferencia de un algoritmo de \textit{tracking} y predicción basados en métodos físicos, prestando atención a los agentes realmente relevantes del escenario de tráfico.

Tras observar las limitaciones de un modelo de predicción basado en cinemática para la predicción a largo plazo de un agente, los distintos algoritmos de la tesis se centran en el módulo de predicción, usando los datasets Argoverse 1 y Argoverse 2, donde se asume que los agentes proporcionados en cada escenario de tráfico ya están monitorizados durante un cierto número de observaciones. 

En primer lugar, se introduce un modelo basado en redes neuronales recurrentes (particularmente redes LSTM, \textit{Long-Short Term Memory}) y mecanismo de atención para codificar las trayectorias pasadas de los agentes, y una representación simplificada del mapa en forma de posiciones finales potenciales en la carretera para calcular las trayectorias futuras unimodales, todo envuelto en un marco GAN (\textit{Generative Adversarial Network}), obteniendo métricas similares al estado del arte en el caso unimodal.

Una vez validado el modelo anterior en Argoverse 1, se proponen distintos modelos base (sólo social, incorporando mapa, y una mejora final basada en \textit{Transformer encoder}, redes convolucionales 1D y mecanismo de atención cruzada para la fusión de características) precisos y eficientes basados en el modelo de predicción anterior, introduciendo dos nuevos conceptos. Por un lado, el uso de redes neuronales gráficas (particularmente GCN, \textit{Graph Convolutional Network}) para codificar de una forma potente las interacciones de los agentes. Por otro lado, se propone el preprocesamiento de trayectorias preliminares a partir de un mapa con un método heurístico. Gracias a estas entradas y una arquitectura más potente de codificación, los modelos base serán capaces de predecir distintas trayectorias futuras multimodales, es decir, cubriendo distintos posibles futuros para el agente de interés. Los modelos base propuestos obtienen métricas de regresión del estado del arte tanto en el caso multimodal como unimodal manteniendo un claro compromiso de eficiencia con respecto a otras propuestas.

El modelo final de la tesis, inspirado en los modelos anteriores y validado en el más reciente dataset para algoritmos de predicción en conducción autónoma (Argoverse 2), introduce varias mejoras para entender mejor el escenario de tráfico y decodificar la información de una forma precisa y eficiente. Se propone incorporar información topológica y semántica de los carriles futuros preliminares con el método heurístico antes mencionado, codificación de mapa basada en aprendizaje profundo con redes GCN, ciclo de fusión de características físicas y sociales, estimación de posiciones finales en la carretera y agregación de su entorno circundante con aprendizaje profundo y finalmente módulo de refinado para mejorar la calidad de las predicciones multimodales finales de un modo elegante y eficiente. Comparado con el estado del arte, nuestro método logra métricas de predicción a la par con los métodos mejor posicionados en el Leaderboard de Argoverse 2, reduciendo de forma notable el número de parámetros y operaciones de coma flotante por segundo.

Por último, el modelo final de la tesis ha sido validado en simulación en distintas aplicaciones de conducción autónoma. En primer lugar, se integra el modelo para proporcionar predicciones a un algoritmo de toma de decisiones basado en aprendizaje por refuerzo en el simulador SMARTS (\textit{Scalable Multi-Agent Reinforcement Learning Training School}), observando en los estudios como el vehículo es capaz de tomar mejores decisiones si conoce el comportamiento futuro de la escena y no solo el estado actual o pasado de esta misma. En segundo lugar, se ha realizado un estudio de adaptación de dominio exitoso en el simulador hiperrealista CARLA en distintos escenarios desafiantes donde el entendimiento de la escena y predicción del entorno son muy necesarios, como una autopista o rotonda con gran densidad de tráfico o la aparición de un usuario vulnerable de la carretera de forma repentina. En ese sentido, el modelo de predicción ha sido integrado junto con el resto de capas de la arquitectura de navegación autónoma del grupo de investigación donde se desarrolla la tesis como paso previo a su implementación en un vehículo autónomo real.

\textbf{Palabras clave:} \myThesisKeywords.

%%% Local Variables:
%%% TeX-master: "../book"
%%% End:


