%%%%%%%%%%%%%%%%%%%%%%%%%%%%%%%%%%%%%%%%%%%%%%%%%%%%%%%%%%%%%%%%%%%%%%%%%%%
%
% Generic template for TFC/TFM/TFG/Tesis
%
% $Id: abstract.tex,v 1.9 2015/06/05 00:10:31 macias Exp $
%
% By:
%  + Javier Macías-Guarasa. 
%    Departamento de Electrónica
%    Universidad de Alcalá
%  + Roberto Barra-Chicote. 
%    Departamento de Ingeniería Electrónica
%    Universidad Politécnica de Madrid   
% 
% Based on original sources by Roberto Barra, Manuel Ocaña, Jesús Nuevo,
% Pedro Revenga, Fernando Herránz and Noelia Hernández. Thanks a lot to
% all of them, and to the many anonymous contributors found (thanks to
% google) that provided help in setting all this up.
%
% See also the additionalContributors.txt file to check the name of
% additional contributors to this work.
%
% If you think you can add pieces of relevant/useful examples,
% improvements, please contact us at (macias@depeca.uah.es)
%
% You can freely use this template and please contribute with
% comments or suggestions!!!
%
%%%%%%%%%%%%%%%%%%%%%%%%%%%%%%%%%%%%%%%%%%%%%%%%%%%%%%%%%%%%%%%%%%%%%%%%%%%

\chapter*{Abstract}
\label{cha:abstract}

\addcontentsline{toc}{chapter}{Abstract}

Autonomous driving is considered one of the greatest technological challenges of today. When autonomous cars take over our roads, accidents will be significantly reduced, almost disappearing, as the technology will be tested and will not violate driving regulations, among other social and economic benefits.

One of the most critical aspects in developing an autonomous vehicle is perceiving and understanding the surrounding scene. This task must be as accurate and efficient as possible to predict the future of the scene and assist in decision-making. In this way, the actions taken by the vehicle will ensure the safety of the vehicle itself and its occupants, as well as that of surrounding obstacles, such as pedestrians, other vehicles, or road infrastructure.

In this context, this doctoral thesis focuses on the study and development of different predictive techniques for scene understanding in the context of autonomous driving. Throughout the thesis, there is a progressive incorporation of Deep Learning techniques into the proposed algorithms to improve reasoning about what is happening in the traffic scenario, as well as modeling the complex interactions between social information (different participants or agents in the scene, such as vehicles, cyclists, or pedestrians) and physical information (geometric, semantic, and topological information from the high-definition map) present in the scene.

The perception layer of an autonomous vehicle is modularly divided into three stages: Detection, Tracking, and Prediction. To start studying the tracking and prediction stages, a Multi-Object Tracking algorithm based on classical motion estimation and association techniques is proposed and validated on the KITTI dataset, which has state-of-the-art metrics. Furthermore, the use of an intelligent filter based on contextual map information is proposed, aiming to monitor the most relevant agents in the scene over time. These filtered agents serve as the preliminary input for making uni-modal predictions based on a kinematic model. In order to validate our proposal including the map-based filter we make use of CARLA (CAR Learning to Act), one of the most promising hyper-realistic simulators for autonomous driving. We demonstrate how using contextual map information can significantly reduce the inference time for physics-based tracking and prediction algorithms by focusing on the truly relevant agents in the traffic scenario.

After observing the limitations of a kinematics-based prediction model for long-term agent prediction, the different algorithms in the thesis focus on the prediction module using the Argoverse 1 and Argoverse 2 datasets. These datasets assume that the agents provided in each traffic scenario have already been monitored for a certain number of observations.

Firstly, a model based on recurrent neural networks (particularly LSTM, Long-Short Term Memory, networks) and an attention mechanism is introduced to encode the past trajectories of agents. It also uses a simplified representation of the map in the form of potential final positions on the road to calculate uni-modal future trajectories. These components are wrapped in a Generative Adversarial Network (GAN) framework, achieving metrics similar to the state-of-the-art in the uni-modal case.

Once the previous model is validated on Argoverse 1, different precise and efficient baseline models are proposed for the prediction module. These baselines (only social, including map and a final improvement based on Transformer encoders, 1D-Convolutional Neural Networks, and cross-attention mechanism for feature fusion) introduce two main concepts. First, we make use of Graph Neural Networks (particularly Graph Convolutional Networks, GCNs) to encode the interactions between agents. Second, we preprocess preliminary trajectories from the map using a heuristic method. These models predict different multi-modal future trajectories, covering various possible futures for the agent of interest. The proposed base models achieve state-of-the-art regression metrics in both multi-modal and uni-modal cases while maintaining a clear efficiency compromise compared to other proposals.

The final model of the thesis, inspired by the previous models and validated on the most recent dataset for autonomous driving prediction algorithms (Argoverse 2), incorporates several improvements to better understand the traffic scenario and decode information accurately and efficiently. It proposes to incorporate topological and semantic information of the preliminary future lanes using the aforementioned heuristic method. It uses map encoding based on Deep Learning with Graph Convolutional Networks, a fusion cycle of physical and social features, Deep Learning estimation of goal points and aggregation of their surrounding environment, and a refinement module to further improve temporal consistency of the final multi-modal predictions in an elegant and efficient manner. Compared to the state-of-the-art, our method achieves prediction metrics on par with the top-ranked methods in the Argoverse 2 Leaderboard while significantly reducing the number of parameters and floating-point operations per second.

Lastly, the final model of this thesis has been validated in simulation in various autonomous driving applications. Firstly, the model is integrated to provide predictions to a Reinforcement Learning-based decision-making algorithm in the SMARTS (Scalable Multi-Agent Reinforcement Learning Training School) simulator, observing in the studies how the vehicle can make better decisions if it knows the future behavior of the scene, not just the current or past state of it. Secondly, a successful domain adaptation study has been conducted in the hyper-realistic CARLA simulator in different challenging scenarios where scene understanding and environment prediction are highly necessary, such as a highway or roundabout with high traffic density or the sudden appearance of a vulnerable road user. In this regard, the prediction model has been integrated alongside the rest of the layers of the autonomous navigation architecture of the research group where the thesis is being developed, as a preliminary step towards its implementation in a real autonomous vehicle.

\textbf{Keywords:} \myThesisKeywordsEnglish.

%%% Local Variables:
%%% TeX-master: "../book"
%%% End:


