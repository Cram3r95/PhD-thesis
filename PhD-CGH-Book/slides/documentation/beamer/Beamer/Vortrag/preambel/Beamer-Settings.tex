% *****************************************
% >>> Themes <<<<<<<<<<<<<<<<<<<<<<<<<<<<<<
% *****************************************

% \usetheme[<options>]{<name list>} 		Installs the presentation theme named <name>.
% \usecolortheme[<options>]{<name list>} Same as \usetheme, only for color themes.
% \usefonttheme[<options>]{<name>} 		Same as \usetheme, only for font themes.
% \useinnertheme[<options>]{<name>}		Same as \usetheme, only for inner themes.
% \useoutertheme[<options>]{<name>}		Same as \usetheme, only for outer themes.

% *****************************************
% >>> Themes ohne Navigation
% *****************************************
% \usetheme{default}
% -------------------------------
% \usetheme[]{Bergen}
% -------------------------------
% \usetheme[%
% 	secheader % section im Header
% ]{Boadilla}
% -------------------------------
% % Wie das Boadilla theme, mit kraeftigeren Farben
% % und unveraenderten Icons
% \usetheme[%
% 	secheader % section im Header
% ]{Madrid}
% -------------------------------
% \usetheme{Pittsburgh}
% -------------------------------
% \usetheme[]{Rochester}


% *****************************************
% >>> Themes mit Navigation (Baumstruktur)
% *****************************************
% \usetheme{Antibes}      % flach
% -------------------------------
% \usetheme{JuanLesPins}  % 3D, Schatten
% -------------------------------
% \usetheme{Montpellier}    % flach, wenig Farben


% *****************************************
% >>> Themes mit Navigation (Sidebar)
% *****************************************
% % flach, starke Farben
% \usetheme[%
%  	left, % sidebar links
% % 	right, % sidebar rechts
% % 	hideallsubsections, % nur sections werden angezeigt
% % 	hideothersubsections, % nur subsections der aktuellen section werden angezeigt
% ]{Berkeley}
% -------------------------------
% % wie Berkeley, 3D, starke Farben
% \usetheme[%
%  	left, % sidebar links
% % 	right, % sidebar rechts
% % 	hideallsubsections, % nur sections werden angezeigt
% % 	hideothersubsections, % nur subsections der aktuellen section werden angezeigt
% ]{PaloAlto}
% -------------------------------
% % flach, schwache Farben
% \usetheme[%
%  	left, % sidebar links
% % 	right, % sidebar rechts
% % 	hideallsubsections, % nur sections werden angezeigt
% % 	hideothersubsections, % nur subsections der aktuellen section werden angezeigt
% ]{Goettingen}
% -------------------------------
% % wie Goettingen, flach, starke Farben
% \usetheme[%
%  	left, % sidebar links
% % 	right, % sidebar rechts
% % 	hideallsubsections, % nur sections werden angezeigt
% % 	hideothersubsections, % nur subsections der aktuellen section werden angezeigt
% ]{Marburg}
% -------------------------------
% flach, schwache Farben
% \usetheme[%
% % 	hideallsubsections, % nur sections werden angezeigt
% % 	hideothersubsections, % nur subsections der aktuellen section werden angezeigt
% ]{Hannover}


% *****************************************
% >>> Themes mit Navigation (Mini Frame Navigation)
% *****************************************
% % starke Farben, flach
% \usetheme[
% % 	compress, % Navigation in einer Zeile
% ]{Berlin}
% -------------------------------
% % wie Berlin, runde Kanten, starke Farben, flach
% \usetheme[
% 	compress, % Navigation in einer Zeile
% ]{Ilmenau}
% -------------------------------
% % wie Berlin, starke Farben, flach
% \usetheme[
% 	compress, % Navigation in einer Zeile
% ]{Dresden}
% -------------------------------
% % starke Farben, 3D
% \usetheme{Darmstadt}
% -------------------------------
% % wie Darmstadt, ohne subsections
% \usetheme{Frankfurt}
% -------------------------------
% % flach, schwache Farben
% \usetheme{Singapore}
% -------------------------------
% % flach, horiz. Linien
% \usetheme{Szeged}


% *****************************************
% >>> Themes mit Navigation (Section and Subsection Tables)
% *****************************************
% % flach, rund, starke Farben
% \usetheme{Copenhagen}
% -------------------------------
% % flach, eckig, starke Farben
% \usetheme{Luebeck}
% -------------------------------
% % flach, starke Farben
% \usetheme{Malmoe}
% -------------------------------
% 3D, starke Farben
\usetheme{Warsaw}


% *****************************************
% >>>> 15.1    Inner Themes
% *****************************************
% An inner theme installs templates that dictate how the following elements are typeset:
%    - Title and part pages.
%    - Itemize environments.
%    - Enumerate environments.
%    - Description environments.
%    - Block environments.
%    - Theorem and proof environments.
%    - Figures and tables.
%    - Footnotes.
%    - Bibliography entries.


% >>>  Itemize Bullets
% -------------------------------
% \useinnertheme{default} % Zahlen
% -------------------------------
% \useinnertheme{circles} % Kreise
% -------------------------------
\useinnertheme{rectangles} % Vierecke
% -------------------------------
% \useinnertheme[%
% 	shadow % mit Schatten
% ]{rounded} % 3D Kugeln
% -------------------------------
% \useinnertheme{inmargin} % Bullts im Margin


% *****************************************
% >>>> 15.2    Outer Themes
% *****************************************
% An outer theme dictates (roughly) the overall layout of frames. It specifies where any navigational elements
% should go (like a mini table of contents or navigational mini frames) and what they should look like. Typically,
% an outer theme specifies how the following elements are rendered:
%    - The head- and footline.
%    - The sidebars.
%    - The logo.
%    - The frame title.

% \useoutertheme{default}
% -------------------------------
% \useoutertheme{infolines}
% -------------------------------
% \useoutertheme[%
% 	footline=empty, % suppressed the footline (default).
% % 	footline=authorinstitute, %shows the author's name and the institute in the footline.
% % 	footline=authortitle, % shows the author's name and the title in the footline.
% % 	footline=institutetitle, % shows the institute and the title in the footline.
% % 	footline=authorinstitutetitle, % shows the author's name, the institute, and the title in the footline.

% ]{miniframes}
% -------------------------------
% \useoutertheme[%
% % 	subsection= true,  % or false shows or suppresses line showing the subsection in the headline.
% ]{smoothbars}
% -------------------------------
% \useoutertheme[%
%  	left, % sidebar links
% % 	right, % sidebar rechts
% % 	hideallsubsections, % nur sections werden angezeigt
% % 	hideothersubsections, % nur subsections der aktuellen section werden angezeigt
% ]{sidebar}
% -------------------------------
% % This theme installs a headline in which, on the left, the sections of the talk are shown and, on the right,
% % the subsections of the current section. If the class option compress has been given, the sections and
% % subsections will be put in one line; normally there is one line per section or subsection.
% \useoutertheme{split}
% The colors are taken from palette primary and palette fourth.
% -------------------------------
% % This layout theme extends the split theme by putting a horizontal shading behind the frame title and
% % adding a little 'shadow' at the bottom of the headline.
% \useoutertheme{shadow}
% -------------------------------
% \useoutertheme[
% % 	hooks, % Einruecken der Abschnittsueberschriften in der Kopfzeile
% ]{tree}
% -------------------------------
% wie tree, ohne die Linien
% \useoutertheme{smoothtree}


% *****************************************
% >>>> 16.1    Color Themes
% *****************************************
% \usecolortheme{default}
% \usecolortheme{structure}
% \usecolortheme{sidebartab}

% *****************************************
% >>>> 16.1.2    Complete Color Themes
% A 'complete' color theme is a color theme that completely specifies all colors for all parts of a frame. It
% installs specific colors and does not derive the colors from, say, the structure beamer-color. Complete
% color themes happen to have names of flying animals.

% -------------------------------
% \usecolortheme{default}
% -------------------------------
\usecolortheme[%
% 	named=red,
% 	named=blue,
% 	named=green,
% 	named=orange,
% 	named=gray,
 	named=NavyBlue,
%  named=RoyalBlue,
%  named=MidnightBlue,
%  named=CadetBlue,
]{structure}
% Farben aus dvipsnam.def: (Option 'dvipsnames' fuer xcolor muss geladen sein!)
% http://www.math.utu.fi/opetusohj/latex/doc/palette.pdf
% GreenYellow, Yellow, Goldenrod, Dandelion, Apricot, Peach, Melon, YellowOrange, Orange, BurntOrange, Bittersweet, RedOrange, Mahogany, Maroon, BrickRed, Red, OrangeRed, RubineRed, WildStrawberry, Salmon, CarnationPink, Magenta, VioletRed, Rhodamine, Mulberry, RedViolet, Fuchsia, Lavender, , Thistle, Orchid, DarkOrchid, Purple, Plum, Violet, RoyalPurple, BlueViolet, Periwinkle, CadetBlue, CornflowerBlue, MidnightBlue, NavyBlue, RoyalBlue, Blue, Cerulean, Cyan, ProcessBlue, , SkyBlue, Turquoise, TealBlue, Aquamarine, BlueGreen, Emerald, JungleGreen, SeaGreen, Greenv,  ForestGreen, PineGreen, LimeGreen, YellowGreen, SpringGreen, OliveGreen, RawSienna, Sepia, Brown, Tan, Gray, Black, White
% -------------------------------
% % blau-schwarz, Hintergrund: blau
% \usecolortheme[%
% %  	overlystylish
% ]{albatross}
% -------------------------------
% % blau-schwarz, Hintergrund: weiss
% \usecolortheme{lily}
% -------------------------------
% % blau-grau, , Hintergrund: grau
% \usecolortheme{beetle}
% -------------------------------
% % gelb-weiss, , Hintergrund: weiss
% \usecolortheme{crane}
% -------------------------------
% grau-grau (hell)
% \usecolortheme{dove}
% -------------------------------
% % grau-grau (dunkel)
% \usecolortheme{fly}
% -------------------------------
% % grau-grau-weiss (hell) mit Boxen
% \usecolortheme{seagull}
% -------------------------------
% % gelb-orange-grau
% \usecolortheme{wolverine}
% -------------------------------
% % grau
% \usecolortheme{beaver}

% *****************************************
% >>>> 16.1.3    Inner Color Themes
% Inner color themes only specify the colors of elements used in inner themes. Most noticably, they specify
% the colors used for blocks. They can be used together with other (color) themes. If they are used to change
% the inner colors installed by a presentation theme or another color theme, they should obviously be specified
% after the other theme has been loaded. Inner color themes happen to have fl‚ower names.
% -------------------------------
% \usecolortheme{lily} % keine Boxen
% -------------------------------
% \usecolortheme{orchid} % Boxen mit starken Farben
% -------------------------------
% \usecolortheme{rose} % Boxen mit schwachen Farben

% *****************************************
% >>>> 16.1.4    Outer Color Themes
% An outer color theme changes the palette colors, on which the colors used in the headline, footline, and
% sidebar are based by default. Outer color themes normally do not change the color of inner elements, except
% possibly for titlelike. They have happen to sea-animal names.

% -------------------------------
% % Titel mit Farbe, starke Farben
% \usecolortheme{whale}
% -------------------------------
% % Titel mit Farbe, schwache Farben
% \usecolortheme{seahorse}
% -------------------------------
% % Titel ohne Farbe, starke Farben
% \usecolortheme{dolphin}

% *****************************************
% Detailierte Veraenderungen der Farben
% *****************************************
% \setbeamercolor*{author in head/foot}{parent=palette tertiary}
% \setbeamercolor*{title in head/foot}{parent=palette secondary}
% \setbeamercolor*{date in head/foot}{parent=palette primary}
% \setbeamercolor*{section in head/foot}{parent=palette secondary} %tertiary
% \setbeamercolor*{subsection in head/foot}{parent=palette primary}
% Elemente deren Farbe veraendert werden kann
% \setbeamercolor{normal text}{fg=black}
% \setbeamercolor*{example text}
% \setbeamercolor*{titlelike}
% \setbeamercolor*{separation line}
% \setbeamercolor*{upper separation line head}
% \setbeamercolor*{separation line}
% \setbeamercolor*{middle separation line head}
% \setbeamercolor*{separation line}
% \setbeamercolor*{lower separation line head}
% \setbeamercolor*{upper separation line foot}
% \setbeamercolor*{middle separation line foot}
% \setbeamercolor*{lower separation line foot}
% -------------------------------
% \setbeamercolor*{math text}
% \setbeamercolor*{math text inlined}
% \setbeamercolor*{math text displayed}
% \setbeamercolor*{normal text in math text}
% -------------------------------
% Nutzung:
% \usebeamercolor[fg]{normal text}
% \setbeamercolor{normal text}{fg=black,bg=mylightgrey}
% -------------------------------
% Palette:
% \setbeamercolor{palette primary}
% \setbeamercolor{palette secondary}
% \setbeamercolor{palette tertiary}
% \setbeamercolor{palette quaternary}
% -------------------------------
% \setbeamercolor{palette sidebar primary}
% \setbeamercolor{palette sidebar secondary}
% \setbeamercolor{palette sidebar tertiary}
% \setbeamercolor{palette sidebar quaternary}




% *****************************************
% Transparenz Effekte
% *****************************************

\setbeamercovered{invisible} % is the default and causes covered text to 'completely disappear.
% \setbeamercovered{transparent} % Durchscheinen des Textes
% \setbeamercovered{dynamic} % Einblenden
% \setbeamercovered{highly dynamic} % Einblenden


% *****************************************
% >>>> 17.1 Font Themes
% *****************************************
% \usefonttheme{default}
% The default font theme installs a sans serif font for all text of the presentation. The default theme
% installs different font sizes for things like titles or head- and footlines, but does not use boldface or
% italics for 'hilighting.' To change some or all text to a serif font, use the serif theme.
% -------------------------------
% \usefonttheme{professionalfonts}
% This font theme does not really change any fonts. Rather, it suppresses certain internal replacements
% performed by beamer. If you use 'professional fonts' (fonts that you buy and that come with a
% complete set of every symbol in all modes), you do not want beamer to meddle with the fonts you use.
% -------------------------------
% \usefonttheme[%
% % 	stillsansserifmath, % mathematical text typeset using sans serif.
% % 	stillsansserifsmall, % will cause 'small' text to be still typeset using sans serif. This refers to
%  								% the text in the headline, footline, and sidebars. Using this options is often
%  								% advisable since small  text is often easier to read in sans serif.
% % 	stillsansseriflarge, %  Titel still   typeset using sans serif
% %  	onlymath, % typset math in serif but nothing else
% ]{serif}
% -------------------------------
% \usefonttheme[%
% % 	onlysmall, % headline, footline, and sidebars is changed
% % 	onlylarge, % main title, frame titles, and section
% ]{structurebold}
% -------------------------------
% \usefonttheme[%
% % 	onlysmall, % headline, footline, and sidebars is changed
% % 	onlylarge, % main title, frame titles, and section
% ]{structureitalicserif}
% -------------------------------
% \usefonttheme[%
% % 	onlysmall, % headline, footline, and sidebars is changed
% % 	onlylarge, % main title, frame titles, and section
% ]{structuresmallcapsserif}



% *****************************************
% >>>> Veraendern der Schrifteinstellung definierter Elemente
% *****************************************

% Beispiele
% \setbeamerfont{frametitle}{size=\large}
% \setbeamerfont{frametitle}{series=\bfseries}

% weitere Befehle
% size= size command sets the size attribute of the beamerfont.
% size*={ size in pt }{ baselineskip }
% shape= (\itshape, \slshape, \scshape, or \upshape)
% series= (command like \bfseries.)
% family= (command like \rmfamily or \sffamily).
% family*={ family name } (For example, the family name for Times happens to be ptm. )
% parent={ parent list } specifies a list of parent fonts.
%
% Example for parent
% \setbeamerfont{parent A}{size=\large}
% \setbeamerfont{parent B}{series=\bfseries}
% \setbeamerfont{child}{parent={parent A, parent B},size=\small}
%
% \usebeamerfont{child}
% This text is small and bold.

% *****************************************
% >>>> 15.3.2    Using Beamer's Templates
% *****************************************
% As a user of the beamer class you typically do not 'use' or 'invoke' templates yourself, directly. For
% example, the frame title template is automatically invoked by beamer somewhere deep inside the frame
% typesetting process. The same is true of most other templates. However, if, for whatever reason, you wish
% to invoke a template yourself, you can use the following command.
% \usebeamertemplate***{ element name }
% -------------------------------
%%% 7.2.1 The Headline and Footline
% \setbeamertemplate{headline} % Beamer-Template/-Color/-Font
% \setbeamertemplate{headline}
% {%
%   \begin{beamercolorbox}{section in head/foot}
%     \vskip2pt\insertnavigation{\paperwidth}\vskip2pt
%   \end{beamercolorbox}%
% }
% \setbeamertemplate{headline}[default] % The default is just an empty headline.
% \setbeamertemplate{headline}[infolines theme]
% \setbeamertemplate{headline}[miniframes theme]
% \setbeamertemplate{headline}[sidebar theme]
% \setbeamertemplate{headline}[smoothtree theme]
% \setbeamertemplate{headline}[smoothbars theme]
% \setbeamertemplate{headline}[tree]
% \setbeamertemplate{headline}[split theme]
% \setbeamertemplate{headline}[text line]{ text } % The headline is typeset with 'text'
% -------------------------------
% \setbeamertemplate{footline} % Beamer-Template/-Color/-Font
% \setbeamertemplate{footline}[default]
% \setbeamertemplate{footline}[infolines theme]
% \setbeamertemplate{footline}[miniframes theme]
% \setbeamertemplate{footline}[page number]
% \setbeamertemplate{footline}[frame number]
% \setbeamertemplate{footline}[split]
% \setbeamertemplate{footline}[text line]{ text }
% -------------------------------
%%% 7.2.2 The Sidebars
% -------------------------------
%%% 7.2.3 Navigation Bars (funktioniert nur mit miniframe Themes)
% \setbeamertemplate{mini frames}[default] % shows small circles as mini frames.
\setbeamertemplate{mini frames}[box] % shows small rectangles as mini frames.
% \setbeamertemplate{mini frames}[tick] % shows small vertical bars as mini frames.
% -------------------------------
%%% 7.2.4 The Navigation Symbols
%%% Beamer-Template/-Color/-Font navigation symbols
\setbeamertemplate{navigation symbols}{} % suppresses all navigation symbols:
% \setbeamertemplate{navigation symbols}[horizontal] % Organizes the navigation symbols horizontally.
% \setbeamertemplate{navigation symbols}[vertical] % Organizes the navigation symbols vertically.
% \setbeamertemplate{navigation symbols}[only frame symbol] % Shows only the navigational symbol for navigating frames.
% -------------------------------
%%% 7.2.5 The Logo
% \setbeamertemplate{logo} % Beamer-Template/-Color/-Font
% -------------------------------
%%% 7.2.6 The Frame Title
% \setbeamertemplate{frametitle} % Beamer-Template/-Color/-Font
% \setbeamertemplate{frametitle}[default][left] % left, center, right
% \setbeamertemplate{frametitle}[shadow theme]
% \setbeamertemplate{frametitle}[sidebar theme]
% \setbeamertemplate{frametitle}[smoothbars theme]
% \setbeamertemplate{frametitle}[smoothtree theme]
% -------------------------------
%%% 7.2.7 The Background
% \setbeamertemplate{background canvas} % Beamer-Template/-Color/-Font
% \setbeamertemplate{background canvas}[default]
% \setbeamertemplate{background canvas}[vertical shading][ color options ] installs a vertically shaded background.
%     - top= color specifies the color at the top of the page. By default, 25% of the foreground of
%       the beamer-color palette primary is used.
%     - bottom= color specifies the color at the bottom of the page. By default, the background of
%       normal text at the moment of invocation of this command is used.
%     - middle= color specifies the color for the middle of the page. Thus, if this option is given, the
%       shading changes from the bottom color to this color and then to the top color.
%     - midpoint= factor specifies at which point of the page the middle color is used. A factor of 0
%       is the bottom of the page, a factor of 1 is the top. The default, which is 0.5 is in the middle.
% \setbeamertemplate{background} % Beamer-Template/-Color/-Font
% \setbeamertemplate{background}[default] % is empty.
% \setbeamertemplate{background}[grid][step=1cm] % places a grid on the background.
%     - step= dimension specifies the distance between grid lines. The default is 0.5cm.
%     - color= color specifies the color of the grid lines. The default is 10% foreground.
% -------------------------------
%%% 7.3 Margin Sizes
\setbeamersize{text margin left=2em,text margin right=2em}
% \setbeamersize{sidebar width left=2cm}
%         - text margin left= TEX dimension sets a new left margin. This excludes the left sidebar. Thus,
%           it is the distance between the right edge of the left sidebar and the left edge of the text.
%         - text margin right= TEX dimension sets a new right margin.
%         - sidebar width left= TEX dimension sets the size of the left sidebar. Currently, this command
%           should be given before a shading is installed for the sidebar canvas.
%         - sidebar width right= TEX dimension sets the size of the right sidebar.
%         - description width= TEX dimension sets the default width of description labels, see Section 11.1.
%         - description width of= text sets the default width of description labels to the width of the
%             text , see Section 11.1.
%         - mini frame size= TEX dimension sets the size of mini frames in a navigation bar. When two
%           mini frame icons are shown alongside each other, their left end points are TEX dimension far
%           apart.
%         - mini frame offset= TEX dimension set an additional vertical offset that is added to the mini
%           frame size when arranging mini frames vertically.
% -------------------------------
%%% 9.1 Adding a Title Page
% \setbeamersize{title page} % Beamer-Template/-Color/-Font
%    This template is invoked when the \titlepage command is used.
%    The following commands are useful for this template:
%     -  \insertauthor inserts a version of the author's name that is useful for the title page.
%     -  \insertdate inserts the date.
%     -  \insertinstitute inserts the institute.
%     -  \inserttitle inserts a version of the document title that is useful for the title page.
%     -  \insertsubtitle inserts a version of the document title that is useful for the title page.
%     -  \inserttitlegraphic inserts the title graphic into a template.
% -------------------------------
%%% 9.2 Adding Sections and Subsections
% -------------------------------
%%% Parent Beamer-Template sections/subsections in toc
% This is a parent template, whose children are section in toc and subsection in toc.
% \setbeamertemplate{sections/subsections in toc}[default]
% \setbeamertemplate{sections/subsections in toc}[sections numbered]
% \setbeamertemplate{sections/subsections in toc}[subsections numbered]
% \setbeamertemplate{sections/subsections in toc}[circle]
\setbeamertemplate{sections/subsections in toc}[square]
% \setbeamertemplate{sections/subsections in toc}[ball]
% \setbeamertemplate{sections/subsections in toc}[ball unnumbered]
% -------------------------------
%%% 9.6 Adding a Bibliography
% -------------------------------
% \setbeamertemplate{bibliography item} % Beamer-Template/-Color/-Font
\setbeamertemplate{bibliography item}[default] %  little article icon as the reference
% \setbeamertemplate{bibliography item}[article] % Alias for the default.
% \setbeamertemplate{bibliography item}[book] % little book icon as the reference
% \setbeamertemplate{bibliography item}[triangle] % triangle as the reference
% \setbeamertemplate{bibliography item}[text] % reference text (like '[Dijkstra, 1982]')
% -------------------------------
%%% 10.1 Adding Hyperlinks and Buttons
% -------------------------------
%%% 11.1 Itemizations, Enumerations, and Descriptions
% \setbeamertemplate{items} % parent template of itemize items and enumerate items
% \setbeamertemplate{itemize items} % Parent Beamer-Template
\setbeamertemplate{itemize items}[triangle]
% \setbeamertemplate{itemize items}[circle]
% \setbeamertemplate{itemize items}[square]
% \setbeamertemplate{itemize items}[ball]
% -------------------------------
% \setbeamertemplate{enumerate items}[default] % Numbered
% \setbeamertemplate{enumerate items}[circle] % Places the numbers inside little circles.
\setbeamertemplate{enumerate items}[square] % Places the numbers on little squares.
% \setbeamertemplate{enumerate items}[ball] % 'Projects' the numbers onto little balls.
% -------------------------------
%%% 11.2 Hilighting
% -------------------------------
%%% 11.3 Block Environments
% \setbeamertemplate{blocks} % Parent Beamer-Template
% \setbeamertemplate{blocks}[default]
\setbeamertemplate{blocks}[rounded][shadow=true]
% \setbeamertemplate{blocks}[rounded][shadow=false]
% -------------------------------
%%% 11.4 Theorem Environments
% \setbeamertemplate{qed symbol} % Beamer-Template/-Color/-Font
% -------------------------------
% \setbeamertemplate{theorems} % Parent Beamer-Template
% \setbeamertemplate{theorems}[default]
% \setbeamertemplate{theorems}[normal font]
% \setbeamertemplate{theorems}[numbered]
% \setbeamertemplate{theorems}[ams style]
% -------------------------------
%%% 11.6 Figures and Tables
% \setbeamertemplate{caption} % Beamer-Template/-Color/-Font
% \setbeamertemplate{caption}[default] typesets the caption name (a word like 'Figure' or 'Abbildung' or 'Table')
% \setbeamertemplate{caption}[numbered] adds the figure or table number to the caption.
% \setbeamertemplate{caption}[caption name own line]
% -------------------------------
% \setbeamertemplate{caption name} % Beamer-Color/-Font
% -------------------------------
%%% 11.10    Abstract
% -------------------------------
%%% 11.11 Verse, Quotations, Quotes
% -------------------------------
%%% 11.12 Footnotes
% -------------------------------
%%% 18.1 Specifying Note Contents
% \setbeamertemplate{note page} % Beamer-Template/-Color/-Font
% \setbeamertemplate{note page}[default]
% \setbeamertemplate{note page}[compress]
% \setbeamertemplate{note page}[plain]
% -------------------------------
%%% Specifying Which Notes and Frames Are Shown
% \setbeameroption{hide notes}
% \setbeameroption{show notes}
% \setbeameroption{show notes on second screen= location }
% \setbeameroption{show only notes}